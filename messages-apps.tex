
\section{Message-passing usecases}
\label{sec:usecases}

In this section we give two applications of the message-passing contract.

\subsection{Memoization}
\label{sec:partial}

There may be strict resource use constraints that apply to executing code on a
blockchain. It may not be possible for a transaction to run the code of a large contract in its
entirety. It may be desirable to divide such code into smaller functions, each requiring
less memory or CPU to run. A script may not trust values pre-computed off-chain,
so a proof that a value was correctly computed on-chain is required.
The technique we describe in this section for constructing such proofs is
similar to a specific kind of caching, also called \emph{memoization} \cite{memoization},
which is also how we refer to our approach.

Consider a function $\fun{myFunction}~:~\mathsf{MyInType}~\to~ \mathsf{MyOutType}$
which performs some computation. We define a script $\mathsf{checkMyFunction}$ \ref{fig:checkmf},
that wraps the computation
done by $\fun{myFunction}$ into a script that produces a message with
data $(\var{fIn},~\var{fOut})$, such that $\fun{myFunction}~\var{fIn}~=~\var{fOut})$,
and mints a message token proving that it is correct :

\begin{figure}[htb]
  \begin{align*}
    \applyScript{\mathsf{checkMyFunction}}~(\wcard,~r,~(\var{tx},~\var{i}))~&\leteq~\\
    &~~~~~\text{\textcolor{gray}{-- If $r$ is a send-message redeemer sending $m$}} \\
    &~~~~~~\text{\textcolor{gray}{--and message data is an input-output pair}} \\
    & \fun{if }~ \fun{fromData}~r~\neq~\emptytype~ \wedge ~\fun{fromData}_{IO}~(\fun{msgData}~m) \neq \emptytype,~\\
    &~~~~~~\text{\textcolor{gray}{-- $m$ is uniquely identified by $i$}} \\
    &~~~~~~\wedge~\fun{inUTxO}~m~=~\outputref~i \\
    &~~~~~~\text{\textcolor{gray}{-- $m$ is from this script}} \\
    &~~~~~~\wedge~ \fun{msgFrom}~m~=~\fun{output}~i \\
    &~~~~~~\text{\textcolor{gray}{-- no minimum sent value}} \\
    &~~~~~~\wedge~\fun{msgValue}~m~=~0 \\
    &~~~~~~\text{\textcolor{gray}{-- message token with message $m$ is minted}} \\
    &~~~~~~\wedge~\msgTkn~m~\subseteq~\mint~\var{tx} \\
    &~~~~~~\text{\textcolor{gray}{-- check myFunction computation}} \\
    &~~~~~~\wedge~\fun{myFunction}~\var{fIn}~=~\var{fOut} \\
    &~~~~~~\where \\
    &~~~~~~~~~~~ ~[(\send,~\Msg~m)] ~=~\fun{fromData}~r \\
    &~~~~~~~~~~~ ~(\var{fIn},~\var{fOut}) ~=~\fun{fromData}_{IO}~(\fun{msgData}~m) \\
    & \fun{else, } \\
    &~~~~~~\False
  \end{align*}
  \caption{Script checking input-output pairs for $\fun{myFunction}$}
  \label{fig:checkmf}
\end{figure}

Note that $\fun{msgTo}$ is not constrained by this contract, so that the generated
message can be sent to any recipient.
The following function decodes the message data as an input-output pair, or fails :
\[ \fun{fromData}_{IO} : \Data \to (\mathsf{MyInType}~\times~ \mathsf{MyOutType})\cup \{\emptytype\} \]

We define a script $\mathsf{useMyFunction}$, Figure \ref{fig:usemf}, that can
consume a message with the redeemer $(\receive,~\Msg~m)$
when it is addressed to the output locked by $\mathsf{useMyFunction}$.
This message serves as a proof that $\fun{myFunction}~(\var{fIn},~\var{fOut})$,
so, $\mathsf{useMyFunction}$ can do some
computation $\fun{checkStuff}$ relying on the fact that $\fun{myFunction}~\var{fIn} = \var{fOut}$.

\begin{figure}[htb]
  \begin{align*}
    \applyScript{\mathsf{useMyFunction}}~&~(d,~r,~(\var{tx},~\var{i}))~\leteq~\\
    &~~~~~\text{\textcolor{gray}{-- If $r$ is a receive-message redeemer sending $m$}} \\
    &~~~~~\text{\textcolor{gray}{-- and $m$-data from message is an input-output pair}} \\
    & \fun{if }~\fun{fromData}~r~\neq~\emptytype~ \wedge ~\fun{fromData}_{IO}~(\fun{msgData}~m) \neq \emptytype,~\\
    &~~~~~\text{\textcolor{gray}{-- $m$ is from the script computing myFunction}} \\
    &~~~~~\wedge~ \fun{msgFrom}~m~=~\mathsf{checkMyFunction} \\
    &~~~~~\text{\textcolor{gray}{-- $m$ is sent to this script}} \\
    &~~~~~\wedge~ \fun{msgTo}~m~=~\fun{output}~i \\
    &~~~~~\text{\textcolor{gray}{-- message token with message $m$ is burned}} \\
    &~~~~~\wedge~(-1)~*~(\msgTkn~m)~\subseteq~\mint~\var{tx} \\
    &~~~~~\text{\textcolor{gray}{-- use myFunction computation output from message to do more stuff}} \\
    &~~~~~\wedge~\fun{checkStuff}~d~r~~(\var{tx},~\var{i})~(\var{fIn},~\var{fOut}) \\
    &~~~~~~\where \\
    &~~~~~~~~~~~ ~[(\receive,~\Msg~m)] ~=~\fun{fromData}~r \\
    &~~~~~~~~~~~ ~(\var{fIn},~\var{fOut}) ~=~\fun{fromData}_{IO}~(\fun{msgData}~m) \\
    & \fun{else, } \\
    &~~~~~\fun{checkOtherStuff}~d~r~~(\var{tx},~\var{i})
  \end{align*}
\caption{Script that uses memoized output of $\fun{myFunction}$}
\label{fig:usemf}
\end{figure}

We give the result that formalizes the use of message-passing here
as proof that $\fun{myFunction}~\var{fIn} = \var{fOut}$.

\paragraph{Lemma (Verified input-output pairs). }
\label{lem:io-pairs}
For any $(s,~u,~\var{tx}, u') \in \LEDGER$, with $\pi~u \neq \emptytype$ and
$(i, ~(\mathsf{useMyFunction}, v, d),~ r) \in \fun{inputs}~\var{tx}$,
such that
\begin{align*}
  [(\receive,~\Msg~m)] &= \fun{fromData}~r \\
  (\var{fIn},~\var{fOut}) ~&=~\fun{fromData}_{IO}~(\fun{msgData}~m) \\
  \fun{msgFrom}~m &= (\mathsf{checkMyFunction}, \wcard, \wcard)
\end{align*}

necessarily $\fun{myFunction}~\var{fIn} = \var{fOut}$, and
$\fun{msgTo}~m = (\mathsf{useMyFunction}, v, d)$.

See \ref{pf:io-pairs} for proof sketch.

% \todopv{properties -  messages are unique, can be exactly one per utxo, all msg tokens
% are in msg utxos}

\subsection{Contracts using message-passing}
\label{sec:async}

Stateful contract interaction, or communication, in the EUTxO model is usually implemented
via dependencies \cite{eutxoma}. A (direct) \emph{dependency} of a script $c$ is a constraint requiring that another
script $c'$ must be executed within the same transaction, possibly with specific
arguments. \emph{Message-passing} is implemented via dependencies, and is an alternative
to the direct dependency approach to implementing interaction and communication
of other contracts or scripts. It allows the interacting scripts execute
asynchronously, and to depend
on the message-passing scripts $\msgsTT$ and $\msgsVal$ rather than
on each other directly. We say that stateful contracts that require the production or
consumption of messages to or from scripts implementing the contract
\emph{use message-passing}, and we formalize this notion in this section.

Messages serve as intermediate stores of assets whose transfer is authorized
by the sender, but before they are accepted by the receiver.
The receiver output script may or may not place additional constraints
on how the value and data in the message is used.
Message-passing is a robust way of
making payouts, which are asset transfers \emph{from} a specific payer \emph{to}
receiver outputs not included in the structured contract
state. We give an example to demonstrate this.

Message-passing specification is closely integrated with ledger semantics,
and inspects the scripts, redeemers, and datums of the input transaction.
Because of this, a message-passing contract must also inspect these
in order to correctly construct a message. So, a state projection function
for a contract that uses message-passing includes the UTxO entry relevant
to the contract state, in full. The contract input must be the complete transaction.
Below, we sometimes abuse notation for predicates $\fun{P} : A \to \B$,
and write $a \in \fun{P}$ for $\fun{P}~a$.

Suppose that $\fun{F} : \Output \mapsto \B$ is some a filter, and
$\fun{c} : \UTxO \to \B$ are some constraints on a valid UTxO state. The contract
denoted by $(\pi_{\fun{F,c}},~\pi_{\Tx},~\STRUC)$ is a structured contract with
\begin{align*}
  \State &\leteq \{ i \mapsto o~\in~U ~\mid~i \in \Input,~o \in \fun{F}, U \in \UTxO, \fun{c}~U \} \\
  \pi_{\fun{F,c}}~ u~&\leteq ~\begin{cases}
    \{ i \mapsto o ~\in u~\mid ~o \in \fun{F} \} & \text{ if } \fun{c}~u \\
    \emptytype  & \text{ otherwise }
  \end{cases}~\\
  \pi_{\Tx} &\leteq \fun{id}
\end{align*}

We can use the technique of
combining structured contracts \cite{structured}, to construct
the structured contract $\STRUC_{\MSGS}$,
\begin{align*}
  \pi_{\State-M}~&\leteq~(\pi_{\fun{F,c}}, \pi_{\Msg}) \\
  \pi_{\Tx-M}~&\leteq~\fun{id}_{\Tx} \\
  \STRUC_{\MSGS} &\leteq \{~ (\emptytype, (s, m), \var{tx}, (s', m'))~\mid~
  (\emptytype, s, \var{tx}, s') \in \STRUC, (\emptytype, m, \var{tx}, m') \in \MSGS~\}
\end{align*}

We call this contract \emph{message-augmentation} of $\STRUC$.
We define the following function that filters messages sent or received by $\STRUC$ :
\begin{align*}
  \fun{getFromSTRUCmsgs} &: \powerset{\Msg} \to \powerset{\Msg}  \\
  \fun{getFromSTRUCmsgs} ~\var{msgs} & \leteq \{~m~\mid~m~\in \var{msgs},~\fun{F}~(\fun{msgFrom}~m)~\} \\
  \nextdef
  \fun{getToSTRUCmsgs} &: \powerset{\Msg}   \to \powerset{\Msg} \\
  \fun{getToSTRUCmsgs} ~\var{msgs} & \leteq \{~m~\mid~m~\in \var{msgs},~\fun{F}~(\fun{msgTo}~m)~\} \\
\end{align*}

We now state a result that says that, for given a $\fun{F,c}$, all messages to and from
the contract $(\pi_{\fun{F, c}},~\pi_{\Tx},~\STRUC)$ for a given step are generated
and consumed only under some appropriate conditions. In particular, messages
\emph{from} a specific contract can only be produced when a script locking
an output of this contract "authorizes" the minting of this message by
successfully validating with a redeemer containing the message(s) being produced.
Consuming messages \emph{to} a specific contract requires, again, the
validation of a script locking the recipient output, given a redeemer
containing these messages.
This result follows directly from the $\MSGS$ specification and implementation.

\paragraph{Lemma ($\STRUC$ messages generated correctly). }
\label{lem:msgs-ok}
For any $(\emptytype, (s, m), \var{tx}, (s' m')) \in \STRUC_{\MSGS}$,
\begin{align*}
  &\forall ~\var{msg} \in \fun{getFromSTRUCmsgs}~(m' \setminus m),
  ((\fun{inUTxO}~\var{msg}) \mapsto (\fun{msgFrom}~\var{msg}) \in s) \\
  &~~~~~ \wedge ~~  ((\fun{inUTxO}~\var{msg}) \mapsto (\fun{msgFrom}~\var{msg}) \notin s') \\
  &~~~~~ \wedge ~~\forall~\var{inp} \in \inputs~\var{tx}, (\outputref~\var{inp} = \fun{inUTxO}~\var{msg} \Rightarrow ~ (\send, \var{msg}) \in \fun{fromData}~(\redeemer~\var{inp}))
  \nextdef
  &\forall ~\var{msg} \in \fun{getToSTRUCmsgs}~(m \setminus m'),
  (\wcard \mapsto (\fun{msgTo}~\var{msg}) \in s) \\
  &~~~~~ \wedge ~~\exists~\var{inp} \in \inputs~\var{tx}, \outputref~\var{inp} = \fun{msgTo}~\var{tx} \wedge (\receive, \var{msg}) \in \fun{fromData}~(\redeemer~\var{inp}))
\end{align*}

\paragraph{Definition (Uses message-passing). }
\label{def:usesmp}
We say that $\STRUC$ \emph{uses message-passing} whenever the set defined by
\[ \fun{getMSGS}~(\emptytype, (s, m), \var{tx}, (s' m'))~\leteq~\fun{getFromSTRUCmsgs}~(m' \setminus m)
\cup \fun{getToSTRUCmsgs}~(m \setminus m')  \]

is non-empty for some $(\emptytype, (s, m), \var{tx}, (s' m')) \in \STRUC_{\MSGS}$. \newline

We define the set of \emph{payouts} in the step $(\emptytype, (s, m), \var{tx}, (s' m')) \in \STRUC_{\MSGS}$ by
\[\fun{getPayouts}~(\emptytype, (s, m), \var{tx}, (s' m'))~\leteq~\{ \var{msg} \in \fun{getFromSTRUCmsgs}~(m' \setminus m) ~\mid ~\fun{msgValue}~\var{msg} > 0 \wedge
\neg~(\fun{F}~(~\fun{msgTo}~\var{msg})) ~\} \]

Whenever this set is non-empty for some step in $\STRUC_{\MSGS}$, we say that it
\emph{makes payouts with messages}.

\paragraph{Discussion}

A contract is said to use message-passing whenever there is a step in $\STRUC_{\MSGS}$ that
requires the production or consumption of a non-empty set of messages to or from $\STRUC$.
Some computations that $\STRUC$ performs may be contingent on receiving a specific
message. For example, accepting a payment message sent by another contract.
 %The lemma \ref{lem:msgs-ok} states

Contracts that use message-passing share two common features that are both necessary
and sufficient for a script $c$ implementing the contract to interface with the message-passing contract :
(i) the script's redeemer must decode to a list of sent/received messages, and (ii)
the script must require that the corresponding messages are minted/burned.
The sent messages must be \emph{from} $c$, and the received messages must be
addressed \emph{to} $c$, by Lemma \ref{lem:msgs-ok}.
% Suppose $\STRUC$
% uses message-passing. Suppose also that
% $(\emptytype, (s, m), \var{tx}, (s' m')) \in \STRUC_{\MSGS}$, and
% $\var{msg} \in \fun{getMSGS}~(\emptytype, (s, m), \var{tx}, (s' m'))$. Then for some
% $(i, ~o,~ r) \in \fun{inputs}~\var{tx}$, with $o \in \fun{F}$, the redeemer $r$
% must decode to a list of sent and received messages, so $\fun{fromData}~r \neq \emptytype$.
% Here, $\fun{fromData}$ is the function that decodes a $\Redeemer$ to the specific
% type $ [(\SR \times \Msg)]$ expected by the script, or fails \ref{fig:msg-to}.

We can refer to the messages sent and received by outputs that make up the state
of a contract $\STRUC$, i.e. those filtered by $\fun{F}$, as a scripts'
\emph{communication}. All other checks and state updates are the contract's
computations. These computations may still include dependencies on scripts implementing
contracts other than $\MSGS$. Specifying when a contract has no non-message
dependencies is important for reasoning about when it is able to progress.
This is, however, the subject of future work.

A \emph{payout} is a message that is from $\STRUC$, but not addressed to $\STRUC$, and
specifies a sent value greater than zero. When a contract makes payouts with messages,
it generates at least one payout for some valid step. Note the function that
returns all the payouts for a given contract, $\fun{getPayouts}$, is a function
of the $\MSGS$ state only, given that it is applied to a valid step of $\STRUC_{\MSGS}$.
To demonstrate using message-passing, we give an example of a message-augmented
contract that makes payouts until it runs out of funds. We focus on a payout
example because we later present a security result about message payouts.

\paragraph{Message-augmented $\PAYOUT$}

Suppose $\fun{NFT}$ is an NFT thread
token whose minting policy requires that a specific output reference must be
spent from the UTxO, and the token must be placed into the output given by
$(\fun{payout}, \fun{NFT} + \fun{a}, \emptytype)$, for some $\fun{a} > 0$.
This NFT token serves as a unique identifier of the UTxO entry currently containing the datum and value
representing part (or all) of the contract state.
This approach is defined in the thread token
design pattern in \cite{eutxoma}, and used in \cite{structured}, as well
as for implementing $\MSGS$.
We define the filter $\fun{F}$ which returns only the outputs containing
the NFT specific to the $\PAYOUT$ contract implementation,
and constraint $\fun{c}$ stating that there is exactly one NFT in the UTxO :
\begin{align*}
  \fun{F}~o~ \leteq~\fun{NFT} \subseteq \val~o
  \nextdef
  \fun{c}~u~\leteq~\exists!~i\mapsto o \in~u, \fun{NFT} \subseteq \val~o
\end{align*}

The redeemer type for the $\fun{payout}$ script is $[(\SR \times \Msg)]$ (as
with all message-passing scripts), and
the datum type is $\{\emptytype\}$, which is never inspected.
We give the two transition rules of the message-augmented specification $\PAYOUT_{\MSGS})$.
The first, $\fun{MSGSOnly}$, applies when the $\PAYOUT$ state is not updated,
and only the message-passing contract is updated. The second, $\fun{PayoutV}$,
applies when a payout of value $\fun{v} > 0$ is made to the address script
$\fun{recipient} \neq \fun{payout}$. Only one payout at a time is possible,
and it must have message index $1$.
Transitions between distinct states with multiple copies of the NFT (e.g.
including more than one UTxO) are never allowed.

We note that rather than defining $\PAYOUT$ first, then $\PAYOUT_{\MSGS}$, we
define the latter directly. The reason for this is that this approach allows
us to express the requirements on the payout messages as constraints on the
$\MSGS$ state update, rather than in terms of on the outputs contained in the
carrying $\var{tx}$. Defining the message-augmentation of any message-passing contract
makes it possible to constrain the message state update, rather than filtering
and constraining messages in the inputs and outputs of a transaction.
This is because the function that
returns all the payouts for a given contract, $\fun{getPayouts}$, is a function
of the $\MSGS$ state only.

% In this case, the constraint specifying when a payout must be
% made, which would be necessary in $\STRUC$,
% \[(\var{tx}, \var{ix}) \mapsto (\fun{payout}, \fun{NFT} + (\var{a}~-~\fun{v}), \emptytype) \in \fun{mkOuts}~\var{tx} \]
%
% is implied by the constraint on the message state update,
% \begin{align*}
%   \var{ms} \leteq (i, 1, (\fun{recipient}, \fun{v}, \emptytype), (\fun{payout}, \fun{NFT} + \var{a}, \emptytype), v, \emptytype)  \\~
%   \var{ms} \in m' ~~\wedge~~ \var{ms} \notin m
% \end{align*}
%
% in the $\STRUC_{\MSGS}$ specification.
% For this contract, the payouts are computed as follows :
%
% \begin{align*}
%   \fun{getPayouts}~(\emptytype, (\{i \mapsto (\fun{payout}, \fun{NFT} + \var{a}, \emptytype)\}, m), \var{tx}, (s' m')) = \begin{cases}
%     (i, 1, (\fun{recipient}, \fun{v}, \emptytype), (\fun{payout}, \fun{NFT} + \var{a}, \emptytype), v, \emptytype)
%     & \text{ if } \exists~\var{inp} \in \inputs~\var{tx}, \outputref~\var{inp} = i   \\
%     \{\} & \text{ otherwise }
% \end{cases}
% \end{align*}

The specification is given by

\begin{equation}
  \inference[MSGSOnly]
  { \\
    \wcard \vdash
    (m)
    \trans{msgs}{\var{tx}}
    (\varUpdate{m'})  &
  (i, o, \wcard) \notin \inputs~\var{tx}\\
  }
  {
  \begin{array}{l}
    \\
  \end{array}
    \vdash
    \left(
    \begin{array}{r}
      \{i \mapsto o\} \\
      m
    \end{array}
    \right)
    \trans{PAYOUT-MSGS}{\var{tx}}
    \left(
    \begin{array}{r}
      \{i \mapsto o\} \\
       \varUpdate{m'}
    \end{array}
    \right) \\
  }
\end{equation}
\nextdef
\begin{equation}
  \inference[PayoutV]
  {~\\
  \var{ms} \leteq (i, 1, (\fun{recipient}, \fun{v}, \emptytype), (\fun{payout}, \fun{NFT} + \var{a}, \emptytype), v, \emptytype)  \\~\\
  \var{ms} \in m' & \var{ms} \notin m &
  \fun{v} \leq \var{a} \\~~\\
  (\var{tx}, \var{ix}) \mapsto (\fun{payout}, \fun{NFT} + (\var{a}~-~\fun{v}), \emptytype) \in \fun{mkOuts}~\var{tx} \\~\\
  \exists~\var{inp} \in \inputs~\var{tx}, \outputref~\var{inp} = i~\wedge~\redeemer~\fun{inp} = \{(\send, \var{ms})\}
  \\~\\
    \wcard \vdash
    (m)
    \trans{msgs}{\var{tx}}
    (\varUpdate{m'}) \\
  }
  {
  \begin{array}{l}
    \\
  \end{array}
    \vdash
    \left(
    \begin{array}{r}
      \{i \mapsto (\fun{payout}, \fun{NFT} + \var{a}, \emptytype)\} \\
      m
    \end{array}
    \right)
    \trans{PAYOUT-MSGS}{\var{tx}}
    \varUpdate{\left(
    \begin{array}{r}
      \{(\var{tx}, \var{ix}) \mapsto (\fun{payout}, \fun{NFT} + (\var{a}~-~\fun{v}), \emptytype)\} \\
       m'
    \end{array}
    \right) } \\
  }
\end{equation}
