\section{Discussion}
\label{sec:discussion}

\subsection{Related work}
\label{sec:related}

In this work we presented a stateful contract implementing asynchronous message-passing
in the EUTxO ledger. Message-passing is the backbone of distributed computing \cite{distributed}
\cite{design}. The $\pi$-calculus process calculus has been developed to formalize
processes and message-passing between them in a distributed computing scenario
\cite{picalc}. We conjecture that it may be possible to apply this formalism to
message-passing between structured contracts, considering them as processes
under certain circumstances. However, this is the subject of future work.

The UTxO ledger design (first introduced as the BitCoin ledger \cite{bitcoin}),
and EUTxO ledger implementations \cite{algorand} \cite{ergo} \cite{nervos} are themselves
message-passing schemes, wherein a transaction is a message to a script.
Our scheme reinterprets messages in a way that allows them to have a single verified
sender output, and a receiver that is also an output. The content of the message
is constrained as well. The contract $\MSGS$
can be viewed as a kind of linear sub-ledger within $\LEDGER$, one that may be
used as a tool in specification and verification of properties, such as temporal
properties of communicating contracts, or ones needed for the applications we presented, e.g.
Lemmas \ref{lem:io-pairs} \ref{lem:msgs-ok}.

Some account-based ledger designs \cite{ethereum} \cite{tezos} \cite{zilliqa}
use on message-passing as the default way of communication between contracts.
However, in all three cases, the message-passing is synchronous, so that both
the sender and receiver accounts are updated as part of processing a single transaction.
Our goal of separating communication from computation for stateful contracts
on the EUTxO ledger is inspired by the Scilla \cite{scilla} programming language.
Even though it was developed for the account-based ledger model, the communicating
automata structure it uses to model contracts could potentially be used to
describe message-passing structured contracts as well.

CoSplit, presented in \cite{sharding}, is a static analysis tool for implementing
\emph{sharding} in an account-based ledger model. Sharding is the practice of
separating contract state into smaller
fragments that can be affected by commuting operations, usually for the purposes of
increasing parallelism and scalability. Our message-passing contract is
distributed, so that each message is contained in a distinct output.
There is no specially-marked output containing the
consolidated message state, whose output reference a transaction must
include every time \emph{any} message is produced or consumed.
So, our design does not put superfluous constraints on the order in which multiple
transactions producing and consuming
messages can be applied to the EUTxO ledger. It does not require further state sharding.

A version of asynchronous message-passing is implemented in the ERC-20 Ethereum contract for
fungible tokens \cite{ethereum-docs}. To transfer an amount of tokens from a sender to a receiver,
the total amount being transferred must first be sent and recorded in an intermediate
data structure, then received and withdrawn from the data structure. The total
amount does not have to be withdrawn in its entirety, which is different from
our design, where a message can only be consumed in full. The ERC-20 design is
also primarily for asset transfers, whereas ours can be used to authenticate
script computation outputs as well. We also note that in an
account-based ledger, transactions interacting with the same stateful
contract like ERC-20 can usually be reordered. However, implementing message-passing
via a centralized data-storage contract in an EUTxO ledger would significantly
reduce the possibility of reordering message-passing transactions.

Formalization of blockchain and ledger functionality forms a foundation for
rigorous reasoning about smart contracts security, discussed in the
detailed overview \cite{leveragingfm}. Mathematical models of
EUTxO and UTxO ledgers and smart contracts therein, including ours, often specify
a simplified version of actual implementations
\cite{eutxoma} \cite{blockalg} \cite{nester} \cite{algorandformal} \cite{bitcoinformal}.
Languages for writing verifiable contracts such as \cite{bitml} \cite{scilla}
make programming on a blockchain ledger more amenable to formal verification.
In this work we continue this tradition of formalizing ledgers, smart contracts, and communication
models between ledger contracts, and formally studying their vulnerabilities. In particular,
this work builds on the work
done in \cite{structured} \cite{eutxoma}
on formalizing stateful program models and their security properties.

\section{Future work}
\label{sec:future}

The scheme we presented in Figure \ref{fig:messages-rule} is such that
the outputs that must be spent in order to
consume a given message are fully specified
(via the $\fun{msgTo}$ field of the message), including their
scripts, values, and datums. In future work, this constraint
could be relaxed for a more permissive system design. For example, the $\fun{msgTo}$
field can be replaced by a constraint on some part of the output. For example,
the recipient output must contain a particular NFT token in its value.
Making it an option to not even specify the receiver allows the message-passing
system to also function as a kind of broadcast system for authenticated data.
Messages broadcast in this way can serve as untrusted oracles.

A \emph{time of expiry} can be added to the message structure and used to specify a time
after which a message can be consumed under constraints other than the
spending of the intended recipient output. This would allow structured contracts
to retract any assets sent via a message but not received by a set time.
Changing the type of the message-passing redeemer from a list of messages
to \emph{either} a list of messages \emph{or} some other data can also
make $\MSGS$ more versatile. It may be useful in
enabling a given script to participate in either computation or communication
as a result of applying a transaction, depending on the redeemer specified.

In this work, we did not specify trace-based properties of $\LEDGER$
or any structured contract $\STRUC$. This topic, in general, is the subject of future
work. Of particular interest are structured contracts that can be guaranteed
to take a step without the need for executing "external" scripts, i.e. ones that
specify permissions for updating parts of the ledger which not reflected in the
contract state representation. It is often not realistic for a contract to
successfully take a step without \emph{any} external contracts validating.
For example, paying into a contract is likely to require an "external" output to be
spent within the same transaction in order to cover the payment. However, it
seems feasible to limit a structured contract's dependencies to message-passing only.
Formalizing and proving properties about this subclass of contracts in the future is
of interest.

The formalization of the double satisfaction problem we presented does not address
the possibility that a script may contain a constraint preventing any other
scripts from executing as part of carrying transaction validation. While this
seems like an obvious way to prevent the problem, we note that "no other
scripts can run during validation of this transaction" is a constraint
on the transaction that is in fact itself vulnerable to double satisfaction.
It is incidental that no other scripts that may benefit from the satisfaction of
this constraint.
This is a special case we can address in the future, however, as mentioned
earlier, it is not likely that a sensible contract can be guaranteed to progress
under the condition that the transaction runs no "external" scripts.

In the future, we intend to mechanize this contract and its applications in Agda,
using the EUTxO ledger and structured contracts model \cite{agda-structured},
as well as in the more realistic Agda-mechanized ledger model \cite{agdaspec} of
the Cardano ledger \cite{cardano}, which is currently in development.

\subsection{Conclusion}
\label{sec:conclusion}

As discussed above, some security properties of
stateful smart contracts in the EUTxO
ledger model have been formalized in existing work.
However, in such models, communication between contracts was not a focus, and was
incidentally treated the same as any other script dependency. In this work, we focus on
formalizing communication of data and assets among
scripts as well as the stateful contracts they implement. We encode communication
as a special kind of dependency of a script (or structured contract) on a distributed
stateful
contract keeping track of all unconsumed messages. Moreover, the contract
we present is specified in the
same small-steps semantic framework as the ledger itself. This allows us to reason about
it separately from other dependencies and from script computation, making it more
amenable to formal reasoning.

As an example of such reasoning, we present
two usecases. The first usecase is a variation on memoization,
wherein message tokens serve as proof
artefacts of successful script computations. The second usecase formalizes
the idea of a script communicating via message-passing, and defines when a message
constitutes a "payout". We go on to formalize the pervasive double satisfaction
problem as a property of a labelled transition system, and demonstrate
the use of message-passing to address this vulnerability in certain cases.



% specifying and reasoning about behaviour of
% stateful programs running on a EUTxO ledger, which we call the structured contract
% formalism. Our formalism defines a robust
% way to relate stateless predicate scripts executed at the ledger level to the
% corresponding corresponding executions of a specific stateful program.
% We used the well-established concept of a subsystem
% to define this relation, and a small-step semantics style already in use in
% existing systems (i.e. Cardano) for the specification of the ledger and contracts.
% This work presents a broadly applicable and principled
% way of reasoning about stateful programs on the EUTxO ledger.
%
% This paper lays the groundwork for treating ledger-implemented programs as
% formal subsystems of the ledger. This is done on a level that is very specific
% to the details of the EUTxO ledger. In the future, we aim to generalize our
% research to be applicable to other kinds of ledger transition systems.
% We would also like to better align our findings with existing concepts in the
% theory of simulation, concurrency, and distributed computation, in order to
% apply the full gamut of results in those areas for studying stateful
% programs on the ledger.
%
% Another goal for future work is to
% find a way to use the findings presented here to verify existing contracts.
% This may be done by first constructing the ledger and transaction representation
% projection functions, then building a state transition specification induced by
% $\LEDGER$ for those projections. Properties of the resulting system can then
% be studied.
%
% A full mechanization in Agda of the results and definitions in this work is currently
% under way, as it is a natural next step. We also intend to build a mechanization of this work
% integrated with the Agda mechanization of a more sophisticated and realistic
% ledger, the Cardano ledger. The Cardano
% ledger is in a unique position to be amenable to the structured contract framework
% approach to verification due to the existence of a mechanized small-steps specification
% in Agda of the entire ledger (currently in development) \cite{agdaspec}.
% Agda automation of proof generation for the subsystem proof obligations, such
% as using an SMT-solver, will be a natural progression of the project.
%
%
% Code errors and design flaws have been very costly for users
% since the introduction of smart contracts on the Ethereum platform \cite{survey}.
% Among the most high-profile and costly being the DAO hack \cite{DAO},
% and more recently, a faulty NFT contract \cite{nftbug}. Formal methods are being
% used to find, prevent, and mitigate vulnerabilities on different ledger models, and via different
% approaches \cite{formal}.
%
%
%
% Scilla \cite{scilla} is a intermediate-level language for writing smart contracts as state
% machines on an account-based ledger model.
% The Scilla authors have used Coq to reason about contracts written in Scilla, proving a variety of
% temporal properties such as safety, liveness, and others. The goal of this work,
% however, not to study stateful program behaviour, but rather, it is to formalize
% the notion of correct implementations
% of stateful programs on a platform where programs are inherently stateless. The purpose of the
% properties we discuss is to exemplify how reasoning about a
% specification trace guarantees the conclusions to hold for any
% ledger representation inducing a correct its implementation. A full treatment
% of lifting safety and liveness properties from specification to implementation
% is the subject of future work. Additionally, we believe that the subsystem approach
% we present in this work may be used to model on-chain interactions between
% Scilla contracts.
%
% The Bitcoin Modelling Language (BitML) \cite{bitml} allows the definition of smart
% contracts running on Bitcoin by means of a restricted class of state machines.
% The BitML state machines are less expressive than the class of specifications
% considered in our model, since we assume that our stateless scripts are written
% in a Turing-complete language. However, the goal of this language is similar to
% ours - to guarantee that the behaviour of certain state machines (in the case of BitML, ones defined
% using this language) if in accordance with the changes made by valid transactions, i.e. soundness.
% The difference is that we present a framework in which one can define a state
% transition system with an implementation (both using a Turing-complete language),
% then prove soundness to achieve a "correct" implementation, whereas BitML allows users to
% define a sound state machine from smaller class, then compile it to a \emph{specific}
% implementation. With BitML, LTL formulas can be automatically
% verified using a dedicated model checker. In the future, we plan to add support for LTL formulas in
% our framework.
%
% VeriSolid \cite{verisolid} synthesises Solidity smart contracts from a state machine specification, and
% verifies temporal properties of the state machine using CTL. The underlying
% ledger model for VeriSolid is, however, an account-based model, rather than the
% EUTxO model we work with. Moreover, in contrast
% to the VeriSolid approach, our approach
% relies on the contract author to themselves to synthesize an implementation
% that meets the requirements specific to the contract being built, and then
% provides a proof obligation to show that implementation is correct. This allows
% for more flexibility in the implementation, as well as in the logic
% used in checking properties. Here, again, it may be of interest to use an approach
% similar to the structured contracts framework to model interactions between
% structured contracts.
%
%
% Our work allows users to compose contracts whose state is distributed across multiple
% UTxOs and tokens on the ledger, and provides a way to formally guarantee that the update of the
% full aggregated state is in accordance with the update of its ledger representation.
% This application of structured contracts serves a similar purpose as for the EUTxO
% ledger as sharding does for an account-based one, as it can be used to increase
% parallelism and scalability. We note that the UTxO model is a natural fit for
% such state separation, since one of the benefits of such a ledger is that
% all operations either commute or fail \cite{parallelism}. Therefore, any
% UTxO state representation,
% with any (correct) implementation, will afford the relevant properties for a given
% distributed contract.
%
% On the EUTxO ledger, the constraint-emitting machine design pattern \cite{eutxoma} makes
% a formal correctness guarantee
% similar to the proof obligation we require as part of the definition of smart contracts.
% However, it is limited to a ledger representation of contract state that is
% strictly the datum and value of a
% single UTxO entry, expressing dependencies on other scripts via a limited set of
% possible constraints on
% transactions. Our model allows the contract state to be computed from multiple
% UTxO entries and tokens aggregated across the ledger state, with its evolution
% coordinated by multiple different scripts. Another notable difference is that
% the ledger model presented in \cite{eutxoma} is a list of UTxO-stype transactions,
% rather than the UTxO set itself, with a unique initial state (the empty ledger).
% Here, we are not able to review and reason about the full transaction history,
% as is the case for existing realistic ledgers.


% Another
% aspirational aspect of automation of this project is that of the translation
% from contract implementations in Agda to Plutus, the language used in the
% between Agda contract implementations and Plutus contracts
