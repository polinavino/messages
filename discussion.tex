\section{Discussion}
\label{sec:discussion}

\subsection{Related work}
\label{sec:related}

Message-passing is the backbone of distributed computing~\cite{distributed,design}.
The $\pi$-calculus process calculus has been developed to formalize
message-passing between processes in distributed computing scenarios~\cite{picalc}.
We conjecture that it may be possible to apply this formalism to
message-passing between structured contracts.

The UTxO ledger model introduced by Bitcoin~\cite{bitcoin},
as well as EUTxO ledger implementations~\cite{ergo},
are themselves message-passing schemes, wherein a transaction is a message to a script.
Our scheme reinterprets messages in a way that allows them to have a single verified
sender output, and a receiver that is also an output. The contract $\MSGS$
can be viewed as a kind of linear sub-ledger within $\LEDGER$, which can be
used as a tool in specification and verification of properties of communicating contracts.

In account-based ledgers~\cite{ethereum,tezos,zilliqa}, (synchronous)
message-passing is the default mode of communication between contracts.
The Scilla programming language~\cite{scilla}, with its emphasis on separating
communication from computation for stateful contracts on the Zilliqa ledger~\cite{zilliqa},
inspired this work.
Even though Scilla was developed for the account-based ledger model, the communicating
automata structure it uses to model contracts may be useful in
describing message-passing structured contracts as well.

Existing work on rigs~\cite{rigs}, which are cryptographic data structures that
provide integrity-at-a-distance, presents an approach to maintaining data
integrity across potentially multiple state-managing machines.
Aspects of this approach are similar in spirit to the thread-token technique
we use to uniquely identify messages and ensure non-duplication of message tokens on
the ledger; both are based on temporal and causal dependencies of operations on one another.

A version of asynchronous, but centralized, message-passing is implemented
in the ERC-20 Ethereum contract for fungible tokens~\cite{erc20}.
The ERC-20 design is primarily for asset transfers, whereas ours can be used to communicate
authenticated data as well. Implementing message-passing
via a centralized data-storage contract such as ERC-20 on an EUTxO ledger would significantly
increase contention over UTxO entries between message-passing transactions, and therefore
reduce concurrency.

Formalization of blockchain and ledger functionality forms a foundation for
rigorous reasoning about smart contracts security, discussed in the
detailed overview~\cite{leveragingfm}. Mathematical models of
EUTxO and UTxO ledgers and smart contracts on those ledgers, including ours,
often specify a simplified version of actual
implementations~\cite{eutxo,eutxoma,blockalg,nester,algorandformal,bitcoinformal}.

% In this work we continue this tradition of formalizing ledgers, smart contracts, and communication
% models between ledger contracts, enabling us to formally study their vulnerabilities. In particular,
% this work builds on the work
% done in \cite{structured} \cite{eutxoma}
% on formalizing stateful program models and their security properties.

\subsection{Future work}
\label{sec:future}

The scheme we presented in Fig.~\ref{fig:messages-rule} is such that
the outputs that must be spent in order to
consume a given message are fully specified
(via the $\fun{msgTo}$ field of the message), including their
scripts, values, and datums. In future work, this constraint
could be relaxed for a more permissive and versatile system design.
A \emph{time of expiry} can be added to the message structure and used to specify a time
after which a message can be consumed under different constraints.
Changing the type of the message-passing redeemer from a list of messages
to a list of messages (for communication) together with some extra data (for computation)
can can allow a given script to more easily engage in both computation and communication
as a result of applying a transaction.

In this work, we did not specify trace-based properties of $\LEDGER$
or any structured contract $\STRUC$. This topic, in general, is the subject of future
work. Of particular interest are structured contracts that can be guaranteed
to take a step without the need for executing "external" scripts, i.e. ones other
than those used to implement that contract. It may be unrealistic for a contract to
always take a step without \emph{any} external contracts validating,
e.g. running a script which locks funds used for paying into the contract. However, it
seems feasible to limit a structured contract's dependencies to message-passing only.
Formalizing and proving properties about this class of structured contracts in the future is
of interest.

In the future, we intend to mechanize this contract and its applications in Agda,
building upon the formal EUTxO ledger model~\cite{eutxo}
and structured contracts framework~\cite{structured}.
To achieve the goal of integrating our work into
the more realistic mechanization of the Cardano ledger~\cite{agdaspec},
and eventually implement it on the platform,
some adjustments to our design may be required.

\subsection{Conclusion}
\label{sec:conclusion}

Principled approaches to implementing and reasoning about the behavior of
stateful smart contracts in the EUTxO
ledger model have already been formalized in existing work.
However, such models do not include any special provisions for analyzing communication
among contracts. In this work, we focus on formalizing communication of data and assets among
scripts as well as the stateful contracts they implement.
We first formalize a common problem in contract interaction --- the double satisfaction problem,
which has to do with a single transaction satisfying the constraints of multiple scripts it executes.
We demonstrate how a single pay-out
made by a transaction in the case where a payment \emph{per executed script}
was expected may constitute an instance of the DS formalism we presented.

We define what a message data structure is in the context of an EUTxO ledger --- a
unique identifier associated with the data and assets being sent, as well as its sender
and receiver outputs. We then define how to use the ledger asset-minting mechanism
to encode messages as tokens which appear in the UTxO set. Messages are
sent and received asynchronously, tracked by a
distributed structured contract $\MSGS$. A proof of correctness of its implementation
guarantees that minted message tokens specify verified sender, receiver, data,
and come with appropriate amounts of sent assets.

To give examples of formal reasoning about the message-passing contract and its applications, we present
two use cases. The first is a variation on memoization,
wherein message tokens specifying input-output pairs of a particular computation
serve as proof artefacts of its correctness. The second formalizes
the idea of structured contract communication via message-passing.
We formalize when a message
constitutes a "payout" from a contract, and then demonstrate
how expressing payouts as messages can address vulnerability to the DSP
in the case of payouts. The necessity of executing the $\MSGS$-implementing
scripts (which may require fee payment) whenever messages are sent and received
is a limitation of our design.
