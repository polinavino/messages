\section{Double Satisfaction}
\label{sec:doublesat}

In the EUTxO model, multiple scripts can be executed as part of the validation and
application of the transaction which causes them to be run, i.e. the \emph{carrying}
transaction. Each script is run when the unique action it is associated with
is performed by the carrying transaction, such as spending a particular UTxO or
minting tokens under a specific
policy. Scripts contain constraints on the data of the carrying transaction,
and \emph{multiple scripts} may place the \emph{same constraint} on the data
of a given transaction. The issue with certain undesirable instances of this situation
is called the \emph{double satisfaction problem} (DSP).

Consider the following examples of constraints a structured contract with
$\Input \leteq \Tx$ may place on a transaction $\var{tx}$ :

\begin{itemize}
  \label{examples}

  \item[(i)] \textbf{Authorization tokens : } the transaction must contain in
  its inputs a special token, the ability to spend which constitutes proof that
  a particular contract state update is authorized

  \item[(ii)] \textbf{Payouts : } to perform a specified state update,
  the transaction must make a payout of value
  $v$ to address $a$, i.e. contain an output with value $v$ and address $a$
  \newline
\end{itemize}

The ability of a transaction author to present an authorization token, as in (i),
is treated as proof that some action the transaction performs is authorized.
It may be that the same transaction makes contract state updates to
other contracts, whose implementations also require the
check that the specific authorization token was presented. There is no problem
with presenting a single authorization token within a transaction updating the
state of multiple distinct contracts, each of which
requires the presentation of this token. This is not an
example of problematic double satisfaction,
since the intended meaning of this constraint is upheld by its implementation.

When two structured contract implementations both place the constraint (ii) on a transaction,
it may be satisfied by a single transaction output $(a, v, \wcard)$. One may
speculate that the authors of each of the
scripts with constraint (ii) intended for the output $(a, v, \wcard)$ to be
somehow associated with the step of the particular contract they care about.
This means that two distinct
outputs would be required, each associated with one contract. In this case,
the double satisfaction is problematic, and $a$ receives less total assets than intended.

Because this issue is one that applies to all scripts
that require transactions to make payouts, it is a widely encountered
issue with EUTxO programming.
It has been discussed in Plutus documentation \cite{plutus-docs}, as well as in the context of
contract audits \cite{marlowe-audit-tech} \cite{ds-blog} \cite{marloweaudit}, but not formally analyzed.
In this work, we discuss the DSP in the context of structured contracts,
and argue that expressing payouts as messages is a way to make
payments that is resilient to DS.

The examples we presented show that the distinction between DS that is problematic
and not problematic is strictly in the \emph{intent of the script author}.
For this reason, we can only judge whether a constraint is \emph{vulnerable}
to double satisfaction, i.e. it is not associated exclusively with a particular
structured contract.
Let us assume that all systems discussed in this section are deterministic,
and define a function that returns all pairs of states in valid transitions
of a contract $\STRUC$ :
\[ \fun{s}~\STRUC = \{~ (s, s')~\mid~\exists~i,~(\emptytype, s, i, s') \in \STRUC~\} \]

We now define what a constraint is, as well as double satisfaction (DS).

\paragraph{Definition (transition constraint). }
A \emph{constraint} of a transition system $\STRUC$ is a subset
\[ \fun{C} \subseteq \emptytypeT \times \State \times \Input \times \State \]

such that $\STRUC \subseteq \fun{C}$. A constraint is \emph{strict} when
$\STRUC \subsetneq \fun{C}$.

\paragraph{Definition (double satisfaction). }
\label{def:ds}
A system $\STRUC$ is \emph{vulnerable to double satisfaction}
with respect to a strict constraint $\fun{C}$
whenever there exists another contract $\STRUC'$, with $\STRUC \subseteq \STRUC'$
and $\fun{s}~\STRUC = \fun{s}~\STRUC'$,

such that $\STRUC \subseteq \STRUC' \cap \fun{C} \subsetneq \STRUC'$.

\paragraph{Discussion. }
Definition \ref{def:ds} states that a double satisfaction-vulnerable constraint
limits the set of allowable inputs for a given pair of states with a transition between them.
This is the kind of constraint that can appear in another contract run within the same
transaction, intuitively making them susceptible to DS. A constraint on the pairs of
states that have a transition between them, on the other hand, can only be satisfied by another
contract if the two contracts \emph{share state}. A constraint on shared contract
state and the update thereof does not appear to be what is meant by
a single constraint being satisfied by multiple contracts. It is, rather, a
different situation, better described in terms of nested or overlapping
contracts, and is the subject of future work.

Our definition of DS rules out constraints that can never be satisfied, and therefore all
constraints that eliminate
certain state transitions entirely. Requiring certain payouts
to be in transaction outputs, but not recording them in the contract state, is
vulnerable to double satisfaction. We give an example to illustrate
the concept of DS vulnerability.

\paragraph{Example ($\TOGGLE$ with extra constraint). }
\label{ex:toggle}
Consider the following specification of a $\TOGGLE$ contract, with $\State = \B$,
and $\Input = (\toggle \cup \{\emptytype\}) \times \Interval{\Tick}$.

\begin{equation}
  \inference[DoNothing]
  { \\
  }
  {
  \begin{array}{l}
    \\
  \end{array}
    \vdash
    \left(
    \begin{array}{r}
      x
    \end{array}
    \right)
    \trans{TOGGLE}{(\emptytype, \wcard)}
    \left(
    \begin{array}{r}
      x
    \end{array}
    \right) \\
  }
\end{equation}
\begin{equation}
  \label{eq:j}
  \inference[Toggle]
  {~\\
  5 \leq k < j \leq 9 \\~\\
  }
  {
  \begin{array}{l}
    \\
  \end{array}
    \vdash
    \left(
    \begin{array}{r}
      x
    \end{array}
    \right)
    \trans{Toggle}{(toggle, [j, k])}
    \varUpdate{\left(
    \begin{array}{r}
      \neg~x
    \end{array}
    \right) } \\
  }
\end{equation}

Now, let us observe the DS vulnerability that occurs in this contract.
We define a contract $\STRUC'$ by removing $j \geq 5$ from \ref{eq:j},
and define the strict constraint
\[ \fun{C} (\emptytype, \wcard, (t, [j,k]), \wcard) \leteq (t = \toggle) \Rightarrow (5 \leq k < j \leq 9) \]

so, $\STRUC \subsetneq \fun{C}$. The contract $\STRUC'$ has the same set of possible
transitions as $\STRUC$, which is : $x \mapsto x$
and $x \mapsto ~\neg~x$, and therefore, $\fun{s}~\STRUC = \fun{s}~\STRUC'$.
So, by our definitions, $\STRUC = \STRUC' \cap \fun{C} \subsetneq \STRUC'$.
Therefore, $\STRUC$ is vulnerable to DS with respect to $\fun{C}$. If such a
vulnerability is deemed problematic by the contract author, then it is likely
that it is important to them that \emph{no other contracts} execute on-chain in the interval
$[5, 9]$. This may be difficult to avoid, in practice.

It is possible to define a class of contracts that is never vulnerable to DS constraints,
and these contracts have an output state for any pair of an input state and an input :

\paragraph{Lemma (DS-free contracts)}
\label{lem:dsfree}

% \begin{itemize}
%   \item[(i)]
  A system $\STRUC$ is not vulnerable to double satisfaction with respect to any
  constraint whenever for any $(s, i)$, there exists
  an $s'$ such that $(\emptytype, s, i, s') \in \STRUC$ \newline
  %
  % \item[(i)] If system $\STRUC$ is not vulnerable to double satisfaction with respect
  % to a constraint $\fun{C}$, it is also not vulnerable to DS with respect to
  % any constraint $\fun{C'} \subseteq \fun{C}$. \newline
  %
  % \item[(iii)] Let $\STRUC$ be contract that uses message-passing, and $\STRUC_m$ -
  % the message-augmented $\STRUC$. Suppose $\fun{getSTRUCmsgs}$ filters the
  % messages such that $\fun{M}$ is a strict constraint of $\STRUC$, as in \ref{def:usesmp}.
  % Then $\STRUC$ is not vulnerable to DS with respect to $\fun{M}$.
% \end{itemize}

We give a proof sketch in \ref{pf:dsfree}.

\paragraph{$\MSGS$ as payouts. }

Payouts may be implemented differently for different structured contracts.
We gave one naive approach to payouts in (ii) in \ref{examples}. This approach is
vulnerable to DS, since the constraint requiring a payout to be made is strictly
on the input transaction, with no record of the payout in the state.
We now argue that, since making payouts via messages can be expressed as
a constraint on a pair of message states, rather than on the input transaction,
this approach to payouts is not vulnerable to DS for a message-enhanced contract.

Let us consider payouts as specified in the definition of payouts via messages,
given in \ref{def:usesmp}.
Suppose $(\pi_{\fun{F,c}},~\pi_{\Tx},~\STRUC)$. Let
 $(\pi_{\fun{F,c}},~\pi_{\Tx},~\STRUC')$ be another structured contract,
with $\fun{s}~\STRUC = \fun{s}~\STRUC'$, and $\STRUC \subseteq \STRUC'$.
The two contracts necessarily have the same sets of payout messages for
transitions between the same pairs of states.

To formalize this reasoning, we define following constraint on the message-enhanced
contract $\STRUC_{\MSGS}$,
\begin{align*}
  \fun{Pays}~(\emptytype, (s, m), \var{tx}, (s' m'))~\leteq~\forall~\STRUC' \supseteq \STRUC,
  \fun{s}~\STRUC = \fun{s}~\STRUC',
  \forall (\emptytype, (s, m), \var{tx'}, (s', m')) \in \STRUC', \\
  ~~~~\fun{getPayouts}_{\STRUC}~(\emptytype, (s, m), \var{tx}, (s', m')) = \fun{getPayouts}_{\STRUC'}~(\emptytype, (s, m), \var{tx'}, (s', m'))
\end{align*}

This constraint states that given any other (more permissive) contract $\STRUC'$,
the set of payouts necessary and sufficient for any step of both $\STRUC_{\MSGS}$ and
$\STRUC'_{\MSGS}$ is fully determined by the change in the corresponding $\MSGS$ state,
even when the input transactions are distinct.
By definition, $\fun{getPayouts}_{\STRUC}$ and
$\fun{getPayouts}_{\STRUC'}$ do not inspect the states $s, s'$ or the
transactions $\var{tx}, \var{tx'}$ when applied to valid steps in $\STRUC$
and $\STRUC'$, respectively, see \ref{sec:async}.
So, we can conclude that the following is trivially true :
\[ \forall ~\var{step}, \fun{Pays}~\var{step} \]

and therefore, $\STRUC' \cap \fun{Pays} = \STRUC'$. We can conclude that payout
messages are not affected
by adding DS-vulnerable constraints of $\STRUC$ to $\STRUC'$.
This gives us the following result, which is a security property
of message-passing contracts :

\paragraph{Lemma ($\MSGS$-payouts and DS)}
Given a filter $\fun{F}$, a UTxO constraint $\fun{c}$, and a structured contract
$(\pi_{\fun{F,c}},~\pi_{\Tx},~\STRUC)$, $\STRUC_{\MSGS}$ is not vulnerable to
DS with respect to the constraint $\fun{Pays}$.
