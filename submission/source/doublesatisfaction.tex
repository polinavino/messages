\section{The problem of double satisfaction}
\label{sec:doublesat}

In the EUTxO model, scripts place constraints on the transactions executing them.
\emph{Multiple scripts} may place the \emph{same constraint} on the data
of a given transaction. The issue with certain undesirable instances of this situation
is called the \emph{double satisfaction problem} (DSP).
The DSP has been discussed in Plutus documentation,\footnote{%
\url{https://plutus.readthedocs.io/en/latest/reference/writing-scripts/common-weaknesses/double-satisfaction.html}
} and in the context of contract audits,\footnote{%
\url{https://medium.com/@vacuumlabs_auditing/cardano-vulnerabilities-1-double-satisfaction-219f1bc9665e}
}\footnote{%
\url{https://github.com/tweag/tweag-audit-reports/blob/main/Marlowe-2023-03.pdf}
}
but has not yet been formally analyzed.

The DSP applies to scripts and structured contracts
that require transactions to make payouts, so it is frequently
encountered in EUTxO script programming. A \emph{naive} payout is a constraint of the
form "the transaction must include an output containing value $v$, with address $a$".
When two structured contract implementations both place such a constraint on a transaction,
it may be satisfied by a single transaction output $(a, v, \_)$, resulting in
insufficient payment made to $a$. Note here that we use
the notation $\_$ to represent a term whose value is not
relevant to the computation in which it appears.

To formalize this kind of vulnerability, let us assume that all systems discussed
in this section are deterministic (i.e. have exactly one end state for each
pair of input and start state),
and define the following function, which returns all pairs of states in all valid transitions
of a given structured contract $\STRUC$ :
\[ \fun{s}~\STRUC = \{~ (s, s')~\mid~\exists~i,~(\emptytype, s, i, s') \in \STRUC~\} \]

\paragraph{Definition (transition constraint). }
A \emph{constraint} of a transition system $\STRUC$ is a subset
$\fun{C} \subseteq \emptytypeT \times \State \times \Input \times \State$
such that $\STRUC \subseteq \fun{C}$.

\paragraph{Definition (double satisfaction). }
\label{def:ds}
A structured contract $(\pi, \pi_{\Tx}, \STRUC)$ is \emph{vulnerable to double satisfaction}
with respect to a constraint $\fun{C}$
whenever there exists another contract $(\pi, \pi_{\Tx}, \STRUC')$, with $\STRUC \subseteq \STRUC'$
and $\fun{s}~\STRUC = \fun{s}~\STRUC'$,
such that $\STRUC' \cap \fun{C} \subsetneq \STRUC'$.

\paragraph{Example ($\TOGGLE$ with extra constraint). }
\label{ex:toggle}
Consider a $(\pi, \pi_{\Tx}, \TOGGLE)$ contract, with $\State \leteq \B$ (i.e. Boolean),
and $\Input \leteq (\toggle \cup \{\emptytype\}) \times \Interval{\Tick}$. \newline


\begin{minipage}{.45\textwidth}
  \begin{equation*}
    \inference[\textsc{NoOp}]
    { \\
    }
    {
    \begin{array}{l}
      \\
    \end{array}
      \vdash
      \left(
      \begin{array}{r}
        x
      \end{array}
      \right)
      \trans{TOGGLE}{(\emptytype, \_)}
      \left(
      \begin{array}{r}
        x
      \end{array}
      \right) \\
    }
  \end{equation*}
\end{minipage}
\begin{minipage}{.45\textwidth}
  \begin{equation*}
    \label{eq:j}
    \inference[\textsc{Toggle}]
    {~\\
    5 \leq j < k \leq 9 \\~\\
    }
    {
    \begin{array}{l}
      \\
    \end{array}
      \vdash
      \left(
      \begin{array}{r}
        x
      \end{array}
      \right)
      \trans{Toggle}{(toggle, [j, k])}
      \varUpdate{\left(
      \begin{array}{r}
        \neg~x
      \end{array}
      \right) } \\
    } \\
  \end{equation*}
\end{minipage}

We define a contract $\STRUC'$ by removing $5 \leq j < k \leq 9$ from rule \textsc{Toggle},
assume it has the same projections $\pi, \pi_{\Tx}$ as $\TOGGLE$,
and define the constraint
\[ \fun{C} (\emptytype, \_, (t, [j,k]), \_) \leteq (t = \toggle) \Rightarrow (5 \leq j < k \leq 9) \]

The contracts $\STRUC'$ and $\STRUC$ transition between the same states:
$\fun{s}~\STRUC = \fun{s}~\STRUC' = {x \mapsto x,\ x \mapsto ~\neg~x}$.
Note that $\STRUC = \STRUC' \cap \fun{C} \subsetneq \STRUC'$, hence
$\STRUC$ is vulnerable to DS with respect to $\fun{C}$.

\bpar{Discussion}
Vulnerability to the DSP comes from the lack of association of some property of
transaction data with a \emph{specific structured
contract} (or script implementing it) that requires this property to hold.
Our DS definition formalizes the association between a transaction property and a
contract by defining a property to be associated with a contract only in the case
when the a property can be expressed as a predicate on the contract state update. Thus,
any property not expressible as a property of a contract state update is vulnerable
to DS.

This defines a broad class of constraints
that are vulnerable to DS, but for which this vulnerability may not necessarily be problematic.
The preceeding example demonstrates an unproblematic constraint, $\fun{C}$, which only requires
that a toggle action happens at a particular time interval. It is possible that multiple
contracts require that some other action be performed in that same time interval
simultaneously.

The onus is on the structured contract author to determine for which constraints placed on the
transaction by the contract double satisfaction vulnerability is a problem.
Then, changes must be made to the contract to ensure that the vulnerability is
removed. A general approach to mitigating arbitrary instances of DSP vulnerability
is outside the scope of this work, however,
in Section~\ref{sec:msgs-pay}, we propose a solution to the DSP for payouts.
A similar approach can be taken for DSP vulnerability for pay-ins.
It is possible to define classes of contracts that are never vulnerable to DS.
See the extended version of this work for a proof of the following lemma:

\paragraph{Lemma (DS-free contracts). }

A deterministic structured contract $(\pi, \pi_{\Tx}, \STRUC)$
is not vulnerable to double satisfaction with respect to any
constraint whenever for any $(s, i)$, there exists
an $s'$ such that $(\emptytype, s, i, s') \in \STRUC$. \newline

\bpar{DSP mitigation}
An existing heuristic for addressing the DSP
is to include a constraint in the implementing script(s)
that forces them to fail if \emph{any other scripts} are being run by the transaction.
This effectively mitigates negative consequences of potential vulnerabilities
of the given contract's constraints to the DSP. This is likely not a practical solution in many
cases, however, as it is too restrictive.
We note that, like the above example, this constraint is not on the
contract state update, but rather,
on the transaction. This means that it is itself vulnerable to DS. However,
vulnerability of this constraint to DS will likely not be deemed to be a problem
by script authors, as the purpose of introducing it is to mitigate the negative consequences of
other constraints' vulnerabilities.
