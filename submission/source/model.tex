\subsection{The \EUTXO ledger model}
\label{sec:sts}

First, we give an overview of the semantics we use for
our contract and ledger specifications,
introduced in prior work~\cite{eutxo,eutxoma,structured},
but included here for the sake of self-containment.

\bpar{Ledger types}
For the purposes of self-containment, we include a description of EUTxO ledger
model types.
We note and justify the (minimal) changes we introduce to the existing model in the description.
The types of booleans, natural numbers, and integers are denoted by
$\B$, $\N$, and $\Z$, respectively. The type $\Ix \leteq \N$ is used for indexing,
e.g. of elements in a list. The type $\Slot \leteq \N$ is used to indicate
blockchain time.

The ledger state consists of a UTxO set,
which is a collection of unspent outputs, each associated with a unique identifier.
An output is a triple
$(a, v, d) \in \Output$, where $a \in \Script$ is the address of the output,
$v \in \Value$ is the collection of assets at this address, and $d \in \Datum$ is
data specified by the user at the time of constructing the transaction which
creates this output (we give more details on these three types below).
The type of
the UTxO set is $\UTxO \leteq \OutputRef \mapsto \Output$, which is
a finite key-value map with unique keys of type $\OutputRef$. We denote a single
element in a finite key-value map $u$ (such as the UTxO set) by $i \mapsto o \in u$.
An output reference $(\var{tx}, \var{ix}) \in \OutputRef \leteq \Tx \times \Ix$ pointing
to an output $o$ in a UTxO set consists of the transaction $\var{tx}$, which
added $(\var{tx}, \var{ix}) \mapsto o$ to the UTxO set, and the index $\var{ix}$,
which is the position of output $o$ in the list of outputs of $\var{tx}$.

The type $\Value$ is used to represent bundles of multiple kinds of assets
Each type of asset in the bundle $v \in \Value$ has a
unique asset ID, $a \in \AssetID \leteq \Policy \times \TokenName$, which
identifies a class of fungible tokens. Associated to the asset ID of each
type of asset in a bundle is a a quantity $q \in \Quantity \leteq \Z$, specifying
the amount of the asset with the given ID in $v$. When $v$ contains $0$ of a
given asset type, its asset ID is not included in $v$.
An asset bundle containing one kind of asset
with asset ID $(\fun{policy},~\fun{tokenName})$
of quantity one is denoted by $\{~\fun{policy} ~\mapsto~\{ \fun{tokenName} \mapsto 1 \}~\}$.

An asset with ID $(p, t)$ has the minting (and burning) policy
$p \in \Policy \leteq \Script$. When an asset under this policy is minted or
burned, the policy script is executed to determine whether the transaction
is allowed to perform this action. The token name $t$ is specified by the user
at the time of constructing the minting transaction. It is used to differentiate
between assets under the same policy. Unlike previous work~\cite{eutxoma,structured},
where the token name is a string, we take $\TokenName \leteq \Data$.
$\Value$ forms a group under addition ($+$) with a zero element ($0$)
and a partial order ($\leq$)~\cite{utxoma}.

A script $s \in \Script$ is a piece of user-defined code that is executed as
part of transaction validation, applied to specific inputs.
Script code is stateless and produces a boolean output.
Scripts are executed as part transaction validation to check
that a transaction is permitted to do the action with which the script is associated.
Scripts are used to specify permissions for two kinds of actions: spending an output (these
are referred to as "validators", or sometimes the "address" of the output), and
minting or burning tokens (these are called minting policies).

We denote script application by
$\applyMPScript{\_}$, followed by the script arguments.
At the time of evaluation,
the arguments supplied to a script consist of transaction data (of the transaction
executing it), as well as the data about the specific action for which
the script specifies permission (i.e. the output being spent, or the tokens
being minted under the policy). An extra piece of data $d \in \Redeemer$, associated
with the particular action being validated, is specified by the user
at the time transaction construction.

On-chain data of variable type, including $\Datum$, $\Redeemer$, and $\TokenName$,
are all type synonyms for $\Data$;
for the sake of brevity,
we will omit explicit calls to the corresponding encoding/decoding functions
as these will be obvious from the types involved,
so any time a value is used as $\Data$ presupposes that decoding is successful.

Updates to the ledger state are specified in the form of a $\Tx$ (transaction) data structure.
A transaction $\var{tx} \in \Tx$
contains (i) a set of \emph{inputs}, each containing an output reference, an output,
and the associated redeemer, (ii) a list of outputs, which get entered into the
UTxO set with the appropriately generated output references, (iii) a pair of slot
numbers representing the validity interval of the transaction, (iv) a $\Value$ being
minted by the transaction, (v) a redeemer for each
of the minting policies being executed, and (vi) the set of (public) keys that signed the
transaction, alongside their signatures.

\bpar{Small-step specifications}
We formulate the transitions of ledgers and contracts
in the form of small-step operational semantics~\cite{plotkin},
as exemplified by the official specification of the Cardano ledger~\cite{shelley,alonzo}.
In our specifications and contract implementation pseudocode,
we follow standard set-theory notation, and clarify any
non-standard notation usage alongside it.

A transition relation $\mathsf{TRANS} \subseteq (\Env \times \State \times \Input \times \State)$
is a collection of 4-tuples. A member $(\var{env},~s,~\var{i},~s')$
of this relation is also denoted by :
  \begin{equation*}
    \var{env} \vdash
    \var{s} \trans{trans}{i} \var{s'}
  \end{equation*}

The variable $\var{env} \in \Env$ is the environment
of the state transition, $s \in \State$ is the starting state, $i \in \Input$ is the input,
and $s' \in \State$ is the end state.
The system $\TRANS$ is a labelled transition system.
For a given transition $(\var{env},~s,~i,~s') \in \TRANS$, the pair $(\var{env},~i)$
of an environment and an input make up the \emph{label} of this transition from $s$ to $s'$.
Conventionally~\cite{shelley}, $\var{env}$ is block-level data, such as blockchain time, whereas
$i$ is specified by the user, e.g. a transaction.

\bpar{Ledger transition semantics}
\label{sec:ledgersem}
The ledger semantics on top of which we build the results of this paper
are found in existing work~\cite{eutxo,eutxoma,structured}, but we include them here
for self-containment and in order to introduce appropriate notation.
The ledger transition system is given by the subset
$\mathsf{LEDGER} \subseteq \Slot \times \UTxO \times \Tx \times \UTxO$.
Membership in this subset is specified by a single transition rule $\fun{ApplyTx}$,
which ensures that
$(\var{slot},~\var{utxo},~\var{tx},~\var{utxo'}) \in \mathsf{LEDGER}$
whenever $\fun{checkTx}~(\var{slot},~\var{utxo},~\var{tx}) = \true$, and $\var{utxo'}$
is given by $(\{~i\mapsto o \in \var{utxo} ~\mid~ i \notin tx.\fun{outputRefs}~\}) \cup ~tx.\fun{outputs} $.
Here, the notation $\fun{r}.f$ represents accessing a (named) field $f$ of a record $\fun{r}$.
This is expressed in rule \textsc{ApplyTx} below,
where any unbound variables are implicitly considered as universally quantified.
\begin{equation*}
\inference[\textsc{ApplyTx}]
{
\var{utxo'}~\leteq~(\{~i\mapsto o \in \var{utxo} ~\mid~ i \notin tx.\fun{outputRefs}~\}) \cup ~tx.\fun{outputs}\\ ~ \\
\fun{checkTx}~(\var{slot},~\var{utxo},~\var{tx})
\\ ~ \\
}
{
\begin{array}{l}
    \var{slot}\\
\end{array}
    \vdash
    \left(
    \begin{array}{r}
    \var{utxo} \\
    \end{array}
    \right)
    \trans{ledger}{tx}
    \left(
    \begin{array}{r}
    \varUpdate{\var{utxo'}}  \\
    \end{array}
    \right) \\
}
\end{equation*}

The function $\fun{checkTx} : \Slot \times \UTxO \times \Tx \to \B$
checks the predicates which are consistent
with the EUTxO model on which this work builds~\cite{eutxoma,structured}.
This includes executing all required validator and minting policy scripts with
the appropriate inputs.
%
The projection $tx.\fun{outputRefs}$ returns a UTxO set containing an entry $k \mapsto o$
for each input $i$ of $\var{tx}$, where the key of
the entry is the output reference $k$ of $i$, and its value
is the output $o$ of $i$.
The value $\var{utxo'}$ is calculated by removing the UTxO entries in $\var{utxo}$
corresponding to those in $tx.\fun{outputRefs}$, and adding the entries
constructed by $tx$, see~\cite{structured} for details. 
