\section{Introduction}
\label{sec:intro}

Message-passing is the standard for communicating data and assets between contracts in
account-based ledgers~\cite{ethereum,zilliqa,tezos}.
An alternative to the account-based ledger model is the EUTxO (extended UTxO) ledger
model, implemented by platforms such as Cardano~\cite{shelley,alonzo}
and Ergo~\cite{ergo}. It is a smart contract- (or \emph{script-}) enabled
UTxO-based ledger model, where user-defined script code is used to specify
conditions a transaction must satisfy to be permitted to spend a UTxO entry or mint a token.
Communication between scripts in EUTxO-based ledgers follows a different architecture
than for account-based models. A
script may require that another script executes successfully within the same transaction. Script interaction
and communication is implemented using these kinds of \emph{script dependencies},
as well as other constraints on the script-executing transaction.

Relying on unstructured, ad-hoc communication among EUTxO scripts presents some
challenges, in particular, in terms of amenability to formal verification.
For example, in order for a stateful contract to make progress,
in addition to executing scripts implementing the contract,
it may be necessary to run arbitrary collections of scripts on which the contract
update depends (either directly or via a sequence of dependencies). Another challenge is
tracking the flow of assets and data, including marking certain quantities of
assets or data as "from" a particular contract (e.g. a payout to a specific address),
or "to" a particular contract
(e.g. a pay-in from an address). Asset flow tracking is a special case of the more
general \emph{double satisfaction problem} (DSP).

The DSP may occur when multiple scripts within
the same transaction share a constraint satisfied by the transaction.
Problematic occurrences of DS are due to the lack of a mechanism to associate the
fulfillment of a constraint with the script imposing said constraint.
For example, a payout is made only once, satisfying two scripts, each of which
was expecting a separate payout.
Because many contracts make payouts, this is a widely discussed
problem in EUTxO programming.
In Section~\ref{sec:doublesat}, we present a formalization of the DSP, which has
not previously been formalized.

Previous work presents principled approaches to building stateful contracts in the
EUTxO ledger model,
such as the constraint-emitting machine design pattern~\cite{eutxo},
as well as the more general structured contract framework (SCF)~\cite{structured}.
%
These have been mechanized in the Agda proof assistant~\cite{agda}
and provide the conceptual basis for the actual specification
of the Cardano ledger specification~\cite{shelley,alonzo},
which is formulated with small-step semantics
and is also mechanized in Agda~\cite{agdaspec}.

We present an EUTxO layer-2 implementation of \emph{asynchronous message-passing} as a
principled approach to communication among individual scripts
and the stateful contracts they implement. Our message-passing architecture is
a stateful contract constructed as an instance of the structured contract
framework.

The state of the message-passing contract is a set of messages.
On the ledger, a special NFT token encodes a single message in the contract state,
and individual message tokens are distributed across distinct UTxO entries.
Each message specifies \emph{sender} and \emph{recipient} UTxO entries, which
are authenticated at the time of minting (burning, resp.) of the message token.
The message token specifies the data being sent, and must be placed in a UTxO also
containing the assets being sent. Any script is able to interface
with the message-passing contract so long as (i) the user input to the script can be
decoded as a list of messages being produced and consumed, and (ii)
the contract ensures the minting and burning of the message NFTs corresponding to
the messages it sends or receives.

Our decentralized stateful contract design constitutes a way to interpret the notion
of message-passing communication on an EUTxO ledger. The scripts implementing
this design
facilitate concurrent updates to asset token balances on the ledger
that exist independently of a shared database.
We argue that the message-passing approach to script and stateful contract communication
addresses some of the challenges of writing scripts to run on an EUTxO ledger.
To that end, we
present two use cases of message-passing together with formal specification and
verification of properties related to the integrity of their behaviour. We demonstrate
how expressing payouts as messages from a stateful contract can resolve the DSP
for payouts. The main contributions of this work are :

\begin{itemize}
\item[(i)] formalization of the double satisfaction problem (Section~\ref{sec:doublesat});
\item[(ii)] specification and implementation of a message-passing structured contract (Section~\ref{sec:messages});
\item[(iii)] an application of the message-passing contract for memoization,
including proven properties of its behaviour (Section~\ref{sec:memo});
\item[(iv)] an application of the message-passing contract as a means of asynchronous
communication of data and assets between structured contracts, which serves as an alternative
to communication via ad-hoc dependencies. We formalize
properties of this application, including resilience of payout messages to the DSP (Section~\ref{sec:async}).
\end{itemize}

We note that an extended version of this paper, containing additional results, proofs,
pseudocode, and examples, is available at \footnote{%
\url{https://fc24.ifca.ai/wtsc/WTSC24_2.pdf}}.
