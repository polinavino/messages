\section{Introduction}
\label{sec:intro}

Message-passing is the standard for communicating data and assets between contracts in
account-based ledgers \cite{ethereum} \cite{zilliqa} \cite{tezos}.
Both sending and receiving a message is performed atomically as part of processing a
transaction containing that message. In some cases, the transaction itself
may be referred to as a message \cite{tezos}.
Communication between scripts in UTxO-based ledgers follows a different architecture.
Smart contract enabled UTxO models such as Cardano \cite{alonzo}, Algorand \cite{algorand-docs},
and Ergo \cite{ergo} use scripts only to \emph{check} that certain changes being
made by a transaction to the ledger state are allowed. A
script may require that another script executes successfully given
certain arguments, and within the same transaction. Script interaction
and communication is implemented via such constraints on other scripts, which we refer to
as \emph{dependencies}.

In this work we present an EUTxO layer-2 implementation of \emph{asynchronous message-passing} as an
alternative method of communication between individual scripts
as well as the stateful contracts they implement (Section \ref{sec:messages}).
We build this infrastructure on top of the extended UTxO ledger presented in
\cite{eutxo}, and formalized in terms of the small-step semantics formalism \cite{steps}
also used to specify the Cardano EUTxO-based ledger \cite{alonzo}. The
stateful message-passing contract is an instantiation of the structured contract
framework \cite{structured}. This framework includes the semantics for specifying
and implementing a stateful contract, together with a proof obligation that
the specification meets the implementation, formalized in Agda \cite{agda-structured}.

The state of the message-passing contract is a set of messages. This state is represented
on the ledger as a collection special NFT tokens storing message data,
which are distributed across separate UTxO entries. Messages specify
the output that was spent to validate their production. Thus, they serve as proof
artefacts of the validation of specific output-locking contracts, with specific
inputs. This gives messages a verified \emph{sender}. Similarly,
message tokens can only be spent and burned by an authenticated \emph{recipient},
another contract that must validate with particular inputs when consuming a message.
Any script is able to interface
with the message-passing contract so long as the user input to the script can be
decoded as a list of messages being produced and consumed.

The message-passing structured contract we propose addresses
some of the challenges of programming on an EUTxO ledger. It does so by
defining a way to record authentically-sourced data and assets on an EUTxO ledger,
for consumption by a specific recipient, in a highly concurrent way.
In addition to providing a way to interpret the notion of message-passing
on an EUTxO ledger, our distributed stateful contract design demonstrates
the flexibility of the structured contract formalism for implementation
of contracts with state contained in multiple UTxO entries. Unlike a centralized
design with a single UTxO entry keeping track of all messages, our design allows
arbitrary reordering of message-passing transactions whenever one of them does not
consume messages (or other outputs) produced by the other.
We demonstrate that this scheme can be used in the following ways, addressing the
corresponding challenges of EUTxO programming :

\begin{itemize}
  \item[(i)] as a authenticated output of a function call with specific
  inputs to the function, similar to the technique called memoization (Section \ref{sec:partial}) ;

  \item[(ii)] as intermediate storage of assets and data for the purpose
  of asynchronous communication between stateful contracts
   (Section \ref{sec:async});
\end{itemize}

Next (Section \ref{sec:doublesat}), we present a formalization of an as-of-yet unformalized, but widely discussed
problem in EUTxO programming :
the \emph{double satisfaction problem} (DSP) \cite{marlowe-audit-tech} \cite{plutus-docs} \cite{ds-blog}
\cite{marloweaudit}. DSP may occur when multiple scripts within
the same transaction share the same constraint, which is satisfied. For example,
by making a \emph{single} payout to an address as required
by two separate contracts, even though the intent of each of the contracts was that
the address should have received a separate payment, for a total of two.
Using the structured contract formalism, we formalize when a constraint
of a contract is vulnerable to such a situation. We then demonstrate that making
payouts via messages is not susceptible to double satisfaction.

The main contributions of this work are :

\begin{itemize}
  \item[(i)] the specification and implementation of a message-passing structured
  contract ;
  \item[(ii)] two applications of the message-passing contract : messages as
  outputs of function calls, and messages-passing as a means of asynchronous
  communication of data and transfer of assets between contracts and within a contract ;
  \item[(iii)] formalization of the double satisfaction problem ;
  \item[(iv)] a result stating that if payouts are specified via changes in the
  state of the message-passing contract, the specification of the required
  payouts is not susceptible to DS
\end{itemize}

% separate computation and communication



% This framework provides a method for
% demonstrating the correctness of the implementation of a stateful program specification.
% Specific design patterns describing a more concrete
% approach to modeling stateful programs on a EUTxO ledger, such as constraint
% emitting machines \cite{eutxoma}, correspond to subsets of the structured
% contracts framework.Our message-passing structured contract,
% supports an asynchronous exchange of messages
%
% specify an asynchronous message-passing structured contract.
% The structured contract framework \cite{structured} is a framework
% for specifying stateful programs running on the EUTxO ledger using
% the same small-steps semantics approach used for specifying ledger
% functionality \cite{alonzo}. This framework provides a method for
% verifying the correctness of the implementation of a stateful program specification.
% Specific design patterns describing a more concrete
% approach to modeling stateful programs on a EUTxO ledger, such as constraint
% emitting machines \cite{eutxoma}, correspond to subsets of the structured
% contracts framework.
%
% Stateful contracts in the extended UTxO model come with a unique
% set of programming challenges. To address some of those challenges,
% in this work we present a structured contract
% formalizing the concept of \emph{message-passing} in the EUTxO ledger.
% It constitutes a new paradigm for managing and implementing asynchronous
% communication between scripts and between structured contracts, as well as between
% scripts implementing a single structured contract.
