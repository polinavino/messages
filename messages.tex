% This is samplepaper.tex, a sample chapter demonstrating the
% LLNCS macro package for Springer Computer Science proceedings;
% Version 2.21 of 2022/01/12
%
\documentclass[runningheads]{llncs}
%
\usepackage[T1]{fontenc}
% T1 fonts will be used to generate the final print and online PDFs,
% so please use T1 fonts in your manuscript whenever possible.
% Other font encondings may result in incorrect characters.
%
\usepackage{graphicx}
% Used for displaying a sample figure. If possible, figure files should
% be included in EPS format.
%
% If you use the hyperref package, please uncomment the following two lines
% to display URLs in blue roman font according to Springer's eBook style:
%\usepackage{color}
%\renewcommand\UrlFont{\color{blue}\rmfamily}
% correct bad hyphenation here
\hyphenation{}

%\usepackage{natbib}
\usepackage{url}


% *** MATHS PACKAGES ***
%
\usepackage{iohk}
%\usepackage[cmex10]{amsmath}
\usepackage{amssymb}
\usepackage{stmaryrd}
%\usepackage{amsthm}

\usepackage[margin=2.5cm]{geometry}
\usepackage{iohk}
\usepackage{microtype}
\usepackage{mathpazo} % nice fonts
\usepackage{amsmath}
\usepackage{amssymb}
%\usepackage{amsthm}
\usepackage{latexsym}
\usepackage{mathtools}
\usepackage{stmaryrd}
\usepackage{extarrows}
\usepackage{slashed}
\usepackage[unicode=true,pdftex,pdfa,colorlinks=true]{hyperref}
\usepackage{xcolor}
\usepackage[capitalise,noabbrev,nameinlink]{cleveref}
\usepackage{float}

% *** ALIGNMENT PACKAGES ***
%
\usepackage{array}
\usepackage{float}  %% Try to improve placement of figures.  Doesn't work well with subcaption package.
\usepackage{subcaption}
\usepackage{caption}

\usepackage{subfiles}
\usepackage{geometry}
\usepackage{listings}
 \usepackage{xcolor}
\usepackage{verbatim}
\usepackage{listings}% http://ctan.org/pkg/listings
\lstset{
  basicstyle=\ttfamily,
  mathescape
}
\usepackage{alltt}
\usepackage{paralist}

% inference rules
\usepackage{semantic}

\usepackage{todonotes}

%
\newcommand{\todochak}[1]{\todo[inline,color=purple!40,author=chak]{#1}}
\newcommand{\todompj}[1]{\todo[inline,color=yellow!40,author=Michael]{#1}}
\newcommand{\todokwxm}[1]{\todo[inline,color=blue!20,author=kwxm]{#1}}
\newcommand{\todopv}[1]{\todo[inline,color=purple!40,author=polina]{#1}}

\newcommand{\red}[1]{\textcolor{red}{#1}}
\newcommand{\redfootnote}[1]{\red{\footnote{\red{#1}}}}
\newcommand{\blue}[1]{\textcolor{blue}{#1}}
\newcommand{\bluefootnote}[1]{\blue{\footnote{\blue{#1}}}}

%% A version of ^{\prime} for use in text mode
\makeatletter
\DeclareTextCommand{\textprime}{\encodingdefault}{%
  \mbox{$\m@th'\kern-\scriptspace$}%
}
\makeatother

\newcommand{\code}{\texttt}
\renewcommand{\i}{\textit}  % Just to speed up typing: replace these in the final version
\renewcommand{\t}{\texttt}  % Just to speed up typing: replace these in the final version
\newcommand{\s}{\textsf}  % Just to speed up typing: replace these in the final version
\newcommand{\msf}[1]{\ensuremath{\mathsf{#1}}}
\newcommand{\mi}[1]{\ensuremath{\mathit{#1}}}

%% A figure with rules above and below.
\newcommand\rfskip{3pt}
%\newenvironment{ruledfigure}[1]{\begin{figure}[#1]\hrule\vspace{\rfskip}}{\vspace{\rfskip}\hrule\end{figure}}
\newenvironment{ruledfigure}[1]{\begin{figure}[#1]}{\end{figure}}

%% Various text macros
\newcommand{\true}{\type{True}}
\newcommand{\false}{\type{False}}

\newcommand{\hash}[1]{\ensuremath{#1^{\#}}}

\newcommand\mapsTo{\ensuremath{\mapsto}}
\newcommand\cL{\ensuremath{\{}}
\newcommand\cR{\ensuremath{\}}}

\newcommand{\List}[1]{\ensuremath{\s{List}[#1]}}
\newcommand{\Set}[1]{\ensuremath{\mathbb{P}~#1}}  %{\ensuremath{\s{Set}[#1]}}
\newcommand{\FinSet}[1]{\ensuremath{\s{FinSet}[#1]}}
\newcommand{\Interval}[1]{\ensuremath{\s{Interval}[#1]}}
\newcommand{\FinSup}[2]{\ensuremath{\s{FinSup}[#1,\linebreak[0]#2]}}
% ^ \linebeak is to avoid a bad line break when we talk about finite
% maps.  We may be able to remove it in the final version.
\newcommand{\supp}{\msf{supp}}

\newcommand{\Script}{\ensuremath{\s{Script}}}
\newcommand{\scriptAddr}{\fun{scriptAddr}}
\newcommand{\ctx}{\ensuremath{\s{Context}}}
\newcommand{\vlctx}{\ensuremath{\s{ValidatorContext}}}
\newcommand{\mpsctx}{\ensuremath{\s{PolicyContext}}}
\newcommand{\toData}{\ensuremath{\s{toData}}}
\newcommand{\toTxData}{\ensuremath{\s{toTxData}}}
\newcommand{\fromData}{\msf{fromData}}

\newcommand{\emptymap}{\ensuremath{\{\}}}
\newcommand{\emptytype}{\star}
\newcommand{\emptytypeT}{\{\star\}}
\newcommand{\verify}{\msf{verify}}

\newcommand{\mkContext}{\ensuremath{\s{mkContext}}}
\newcommand{\mkTxInfo}{\ensuremath{\s{mkTxInfo}}}
\newcommand{\mkVlContext}{\ensuremath{\s{mkValidatorContext}}}
\newcommand{\mkMpsContext}{\ensuremath{\s{mkPolicyContext}}}
\newcommand{\checkSig}{\ensuremath{\s{checkSig}}}

\newcommand{\applyScript}[1]{\ensuremath{\llbracket#1\rrbracket}}
\newcommand{\applyMPScript}[1]{\ensuremath{\llbracket#1\rrbracket}}

\newcommand{\unionoverrideMinus}{\ensuremath{\mathbin{\cup_{-}}}}
% wildcard parameter
\newcommand{\wcard}[0]{\underline{\phantom{a}}}

% Macros for eutxo things.
\newcommand{\tx}{\fun{tx}}
\newcommand{\TxId}{\ensuremath{\s{TxId}}}
\newcommand{\TxInfo}{\ensuremath{\s{TxInfo}}}
\newcommand{\txId}{\msf{txId}}
\newcommand{\txrefid}{\fun{id}}
\newcommand{\Address}{\ensuremath{\s{Address}}}
\newcommand{\DataHash}{\ensuremath{\s{DataHash}}}
\newcommand{\hashData}{\fun{dataHash}}
\newcommand{\idx}{\fun{index}}
\newcommand{\inputs}{\fun{inputs}}
\newcommand{\outputs}{\fun{outputs}}
\newcommand{\Out}{\type{Output}}
\newcommand{\validityInterval}{\fun{validityInterval}}
\newcommand{\scripts}{\fun{scripts}}
\newcommand{\mint}{\fun{mint}}
\newcommand{\mintScripts}{\fun{mintScripts}}
\newcommand{\mintScsRdmrs}{\fun{mintScsRdmrs}}
\newcommand{\mintRdmrs}{\fun{mintRdmrs}}
\newcommand{\sigs}{\fun{sigs}}
\newcommand{\fee}{\fun{fee}}
\newcommand{\addr}{\fun{addr}}
\newcommand{\pubkey}{\fun{pubkey}}
\newcommand{\val}{\fun{value}}  %% \value is already defined
\newcommand{\Value}{\type{Value}}
\newcommand{\Redeemer}{\type{Redeemer}}
\newcommand{\TxOutRef}{\type{TxIn}}
\newcommand{\TxOut}{\type{TxOut}}
\newcommand{\ScriptContext}{\type{ScriptContext}}
\newcommand{\ScriptPurpose}{\type{ScriptPurpose}}
\newcommand{\Datum}{\type{Datum}}
\newcommand{\DCert}{\type{DCert}}
\newcommand{\LCTx}{\type{LCTx}}
\newcommand{\TxInInfo}{\type{TxInInfo}}

\newcommand{\FState}{\type{FState}}
\newcommand{\FInput}{\type{FInput}}
\newcommand{\Send}{\type{Send}}
\newcommand{\Receive}{\type{Receive}}
\newcommand{\SimplInput}{\type{SimplInput}}
\newcommand{\AccMsgs}{\type{AccMsgs}}
\newcommand{\Context}{\type{Context}}

\newcommand{\validator}{\fun{validator}}
\newcommand{\redeemer}{\fun{redeemer}}
\newcommand{\datum}{\fun{datum}}
\newcommand{\datumHash}{\fun{datumHash}}
\newcommand{\datumWits}{\fun{datumWitnesses}}
\newcommand{\Data}{\ensuremath{\s{Data}}}
\newcommand{\Input}{\ensuremath{\s{Input}}}
\newcommand{\Output}{\type{Output}}
\newcommand{\OutputRef}{\fun{OutputRef}}
\newcommand{\Signature}{\ensuremath{\s{Signature}}}
\newcommand{\Ledger}{\ensuremath{\s{Ledger}}}

\newcommand{\outputref}{\fun{outputRef}}
\newcommand{\outputrefs}{\fun{outputRefs}}
\newcommand{\txin}{\fun{in}}
\newcommand{\id}{\fun{id}}
\newcommand{\lookupTx}{\msf{lookupTx}}
\newcommand{\getSpent}{\msf{getSpentOutput}}

\newcommand{\Tick}{\ensuremath{\s{Slot}}}
\newcommand{\currentTick}{\msf{currentTick}}
\newcommand{\spent}{\msf{spentOutputs}}
\newcommand{\unspent}{\msf{unspentOutputs}}
\newcommand{\txunspent}{\msf{unspentTxOutputs}}
\newcommand{\eutxotx}{\msf{Tx}}


\newcommand{\consumes}[1]{\msf{consumes(#1)}}
\newcommand{\consumesOne}[1]{\msf{consumesOne(#1)}}
\newcommand{\cid}{\fun{cid}}
\newcommand{\inputValue}{\fun{inputValue}}
\newcommand{\rMin}{r_{\fun{min}}}
\newcommand{\rMax}{r_{\fun{max}}}

\newcommand{\utxotx}{\msf{Tx}}

\newcommand{\Quantity}{\ensuremath{\s{Quantity}}}
\newcommand{\Asset}{\ensuremath{\s{Asset}}}
\newcommand{\Policy}{\ensuremath{\s{PolicyID}}}
\newcommand{\Quantities}{\ensuremath{\s{Quantities}}}
\newcommand{\nativeCur}{\ensuremath{\mathrm{nativeC}}}
\newcommand{\nativeTok}{\ensuremath{\mathrm{nativeT}}}
\newcommand{\valC}{\mkValidator{\mathcal{C}}}
\newcommand{\polC}{\mkPolicy{\mathcal{C}}}
\newcommand\mkValidator[1]{\msf{validator}_#1}
\newcommand\mkPolicy[1]{\msf{policy}_#1}

\newcommand{\PublicKey}{\ensuremath{\s{PubKey}}}
\newcommand{\PubKey}{\ensuremath{\s{PubKey}}}
\newcommand{\PrivateKey}{\ensuremath{\s{PrivateKey}}}

\newcommand{\pkey}{\ensuremath{\pi_{\mathsf{p}}}}
\newcommand{\skey}{\ensuremath{\pi_{\mathsf{s}}}}

\newcommand\B{\ensuremath{\mathbb{B}}}
\newcommand\N{\ensuremath{\mathbb{N}}}
\newcommand\Z{\ensuremath{\mathbb{Z}}}
\renewcommand\H{\ensuremath{\mathbb{H}}}
%% \H is usually the Hungarian double acute accent
\newcommand{\emptyBs}{\ensuremath{\emptyset}}
\newcommand{\leteq}{\ensuremath{\mathrel{\mathop:}=}}
\newcommand{\Nt}{\ensuremath{\Diamond}}
\newcommand{\Bool}{\type{Bool}}
\newcommand{\Type}{\type{Type}}
\newcommand{\STRUC}{\type{STRUC}}
\newcommand{\BASIC}{\type{BASIC}}
\newcommand{\LEDGER}{\type{LEDGER}}
\newcommand{\LChanges}{\type{LChanges}}
\newcommand{\LVal}{\type{LVal}}

%ledger spec commands
\newcommand{\Request}{\type{Request}}
\newcommand{\LS}{\mathcal{L}\mathcal{S}}
\newcommand{\State}{\type{State}}
\newcommand{\CEMState}{\type{CEMState}}
% \newcommand{\AccID}{\type{AccID}}
\newcommand{\PCMT}{\type{PCMT}}
\newcommand{\Accts}{\type{Accts}}
% \newcommand{\Open}{\type{Open}}
% \newcommand{\OArgs}{\type{OArgs}}
% \newcommand{\Close}{\type{Close}}
% \newcommand{\CArgs}{\type{CArgs}}
% \newcommand{\Deposit}{\type{Deposit}}
% \newcommand{\DArgs}{\type{DArgs}}
\newcommand{\AMState}{\type{AMState}}
% \newcommand{\Withdraw}{\type{Withdraw}}
% \newcommand{\WArgs}{\type{WArgs}}
% \newcommand{\Transfer}{\type{Transfer}}
\newcommand{\TransferFrom}{\type{TransferFrom}}
\newcommand{\TransferTo}{\type{TransferTo}}
% \newcommand{\TArgs}{\type{TArgs}}
\newcommand{\TTArgs}{\type{TTArgs}}
\newcommand{\TFArgs}{\type{TFArgs}}
\newcommand{\CEMInput}{\type{CEMInput}}
\newcommand{\Tx}{\type{Tx}}
\newcommand{\ID}{\type{ID}}
\newcommand{\Th}{\mathcal{T}}
\newcommand{\Tu}{\mathcal{U}}
\newcommand{\T}{\type{T}}
\newcommand{\Err}{\type{Err}}
\newcommand{\ups}{\fun{update}}
\newcommand{\FHBMT}{\type{FHBMT}}
\newcommand{\UTxO}{\type{UTxO}}
\newcommand{\UTxOState}{\type{UTxOState}}
\newcommand{\TxIn}{\type{TxIn}}
\newcommand{\MsgIn}{\type{MsgIn}}

\newcommand{\CEMInit}{\type{CEMInit}}
\newcommand{\CEMStep}{\type{CEMStep}}

\newcommand{\CEMInputs}{\type{CEMInputs}}
\newcommand{\CEMStates}{\type{CEMStates}}

\newcommand{\PlutusVII}{\type{PlutusV2}}
\newcommand{\Credential}{\type{Credential}}
\newcommand{\Block}{\type{Block}}

\newcommand{\StateMachineState}{\type{StateMachineState}}
\newcommand{\StateMachineInput}{\type{StateMachineInput}}
\newcommand{\NFTSMState}{\type{NFTSMState}}
\newcommand{\NFTSMInput}{\type{NFTSMInput}}

\newcommand{\True}{\type{True}}
\newcommand{\False}{\type{False}}

\newcommand{\Ix}{\type{Ix}}
\newcommand{\Slot}{\type{Slot}}
\newcommand{\PParams}{\type{PParams}}
\newcommand{\Coin}{\type{Coin}}
\newcommand{\LState}{\type{LState}}
\newcommand{\Initialize}{\type{Initialize}}
\newcommand{\Progress}{\type{Progress}}

\newcommand{\OutputVal}{\type{OutputVal}}

\newcommand{\AssetID}{\type{AssetID}}
\newcommand{\TokenName}{\type{TokenName}}
\newcommand{\myThread}{\type{myThread}}
\newcommand{\pkSigns}{\type{pkSigns}}

\newcommand{\Counters}{\type{Counters}}
\newcommand{\CounterID}{\type{CounterID}}
\newcommand{\CS}{\type{CS}}
\newcommand{\CI}{\type{CI}}

\newcommand{\Env}{\type{Env}}
\newcommand{\FEnv}{\type{FEnv}}
\newcommand{\Id}{\type{Id}}

\usepackage{etoolbox}
\usepackage{tikz-qtree}
\usepackage{tikz}
\usetikzlibrary{matrix}

% For anonymisation
\newtoggle{anonymous}
\toggletrue{anonymous}
\iftoggle{anonymous}{
  \newcommand{\Cardano}{CHAIN}
  \newcommand{\Plutus}{LANG}
}{
  \newcommand{\Cardano}{Cardano}
  \newcommand{\Plutus}{Plutus Core}
}

% Names, for consistency
\newcommand{\UTXO}{UTxO}
\newcommand{\EUTXO}{E\UTXO{}}
\newcommand{\ExUTXO}{Extended \UTXO{}}

\newcommand{\SELF}{\type{SELF}}
\newcommand{\TRANS}{\type{TRANS}}
\newcommand{\POV}{\type{POV}}
\newcommand{\CEM}{\type{CEM}}
\newcommand{\Trc}{\type{Trc}}
\newcommand{\CEMS}{\type{CEMS}}
\newcommand{\SMUP}{\type{SMUP}}
\newcommand{\CEMsm}{CEM_{sm}}

\newcommand{\SR}{\type{SR}}
\newcommand{\MsgRdmr}{\type{MsgRdmr}}
\newcommand{\send}{\type{send}}
\newcommand{\receive}{\type{receive}}
\newcommand{\MSGS}{\type{MSGS}}
\newcommand{\Msg}{\type{Msg}}
\newcommand{\msgsTT}{\type{msgsTT}}
\newcommand{\msgsVal}{\type{msgsVal}}
\newcommand{\msgTkn}{\type{msgTkn}}

\newcommand{\Memory}{\type{Memory}}
\newcommand{\Params}{\type{Params}}
\newcommand{\HState}{\type{HState}}
\newcommand{\SType}{\type{SType}}
\newcommand{\HYDRA}{\type{HYDRA}}

\newcommand{\TOGGLE}{\type{TOGGLE}}
\newcommand{\Toggle}{\type{Toggle}}
\newcommand{\toggle}{\type{toggle}}
\newcommand{\Payout}{\type{Payout}}
\newcommand{\payout}{\type{payout}}
\newcommand{\myPKs}{\type{myPKs}}
\newcommand{\toggleTT}{\type{toggleTT}}
\newcommand{\toggleVal}{\type{toggleVal}}
\newcommand{\PartialOutput}{\type{PartialOutput}}

\newcommand{\UTXOma}{\UTXO$_{\textrm{ma}}$}
\newcommand{\EUTXOma}{\EUTXO$_{\textrm{ma}}$}

\newcommand\initial{\msf{initial}}
\newcommand\nft{\blacklozenge}
\newcommand\step{\msf{step}}
\newcommand\satisfies{\msf{satisfies}}
\newcommand\checkOutputs{\msf{checkOutputs}}
\newcommand\txeq{tx^\equiv}

% Account simulations
\newcommand{\threadTT}{\type{threadTT}}
\newcommand{\AccID}{\type{AccID}}
\newcommand{\AccData}{\type{AccData}}
\newcommand{\Accnts}{\type{Accnts}}
\newcommand{\OneUpd}{\type{OneUpd}}
\newcommand{\AccUpd}{\type{AccUpd}}
\newcommand{\Open}{\type{Open}}
\newcommand{\Oargs}{\type{Oargs}}
\newcommand{\Close}{\type{Close}}
\newcommand{\CArgs}{\type{CArgs}}
\newcommand{\Deposit}{\type{Deposit}}
\newcommand{\DArgs}{\type{DArgs}}
\newcommand{\Withdraw}{\type{Withdraw}}
\newcommand{\WArgs}{\type{WArgs}}
\newcommand{\Transfer}{\type{Transfer}}
\newcommand{\TArgs}{\type{TArgs}}
\newcommand{\ACCNTS}{\type{ACCNTS}}
\newcommand{\ACCOP}{\type{ACCOP}}
\newcommand{\PAYOUT}{\type{PAYOUT}}

% relaxed float placement
\renewcommand{\topfraction}{.95}
\renewcommand{\bottomfraction}{.7}
\renewcommand{\textfraction}{.15}
\renewcommand{\floatpagefraction}{.66}
\renewcommand{\dbltopfraction}{.66}
\renewcommand{\dblfloatpagefraction}{.66}
\setcounter{topnumber}{9}
\setcounter{bottomnumber}{9}
\setcounter{totalnumber}{20}
\setcounter{dbltopnumber}{9}

\newcommand\CStep[1]{\ensuremath{
 #1 \xrightarrow{\hspace{5pt} i \hspace{5pt}} (#1' , \txeq)
%%\textsf{step}\, #1\, i \equiv \textsf{just}\, #1'
}}

\newcommand{\pp}{\nolinebreak\hspace{+0.3em}{\tiny\bf +}\nolinebreak\hspace{-.10em}{\tiny\bf +}\nolinebreak\hspace{+0.3em}\nolinebreak}


\begin{document}
%
\title{Message-passing and the Double Satisfaction Problem in the EUTxO Ledger Model \thanks{This work was supported
by Input Output (iohk.io) through their funding of the Edinburgh Blockchain Technology Lab.}}
%
%\titlerunning{Abbreviated paper title}
% If the paper title is too long for the running head, you can set
% an abbreviated paper title here
%

\author{Polina  Vinogradova\inst{1}\orcidID{0000-0003-3271-3841} }
% \and
% Manuel Chakravarty\inst{1}\orcidID{0000-1111-2222-3333} \and %}{IOG} {manuel.chakravarty@iohk.io}{}{}
% Philip Wadler\inst{1,2}\orcidID{0000-1111-2222-3333} \and
% James	Chapman\inst{1}\orcidID{0000-1111-2222-3333} \and %}{IOG} {james.chapman@iohk.io}{}{}
% Tudor	Ferariu\inst{1,2}\orcidID{0000-1111-2222-3333} \and %}{University of Edinburgh, UK} {s1408714@sms.ed.ac.uk}{}{}
% Michael Peyton Jones\inst{1}\orcidID{0000-1111-2222-3333} \and %}{IOG} {michael.peyton-jones@iohk.io}{}{}
% Jacco	Krijnen\inst{1,3}\orcidID{0000-1111-2222-3333} \and %}{Utrecht University, Netherlands} {j.o.g.krijnen@uu.nl}{}{}
% Orestis	Melkonian\inst{1,2}\orcidID{0000-1111-2222-3333} } %}{IOG \& University of Edinburgh, UK} {orestis.melkonian@ed.ac.uk}{}{}
%
\authorrunning{ P. Vinogradova } % et al.}
% \title{Designing EUTxO smart contracts as communicating state machines: the case of simulating accounts}
%
% \titlerunning{Designing EUTxO smart contracts as communicating state machines} %TODO optional, please use if title is longer than one line
%

% First names are abbreviated in the running head.
% If there are more than two authors, 'et al.' is used.
%
\institute{
  IOG,
  \email{firstname.lastname@iohk.io}
  % \and
  % University of Edinburgh, UK
  % \email{orestis.melkonian@ed.ac.uk, s1408714@sms.ed.ac.uk, wadler@inf.ed.ac.uk}
  % \and
  % Utrecht University, Netherlands,
  % \email{j.o.g.krijnen@uu.nl}
}
% \email{lncs@springer.com}\\
% \url{http://www.springer.com/gp/computer-science/lncs} \and
% ABC Institute, Rupert-Karls-University Heidelberg, Heidelberg, Germany\\
% \email{\{abc,lncs\}@uni-heidelberg.de}}
%
\maketitle              % typeset the header of the contribution
%
% An EUTxO ledger is a smart contract-enabled UTxO ledger with native
% multi-asset support \cite{eutxoma}. Unlike account-based ledgers, an EUTxO ledger
% does not have built-in support for message-passing.

% Like account-based messages, our
% design validates the sender (receiver) scripts of a message at the time of
% production (reps. consumption).

\begin{abstract}
  We formalize communication via message-passing among scripts and stateful contracts in
  the extended UTxO model. Our formalization is made up of a specification and an
  implementation of a message-passing
  stateful smart contract using the structured contracts framework.
  Messages are recorded on the ledger in the form of special NFT tokens, distributed one
  per UTxO entry. This entry also contains the assets the message is sending.
  We formalize the notion of sender and receiver outputs for messages.
  A message is sent (produced) and received (consumed)
  asynchronously, in separate transactions, with the possibility of multiple messages
  between multiple outputs to be produced and consumed within a single transaction.

  We give two applications of our
  design. The first is using a message as a record of a successful script computation,
  similar to memoization.
  The second application is as an mechanism for asynchronous communication within a
  structured contract, as
  well as externally, such as in the case of making payouts to an address, separating
  contract communication from computation.
  We also formalize the well-known double satisfaction problem for the EUTxO
  ledger, which has not been done before. We then show how message-passing
  can solve the double satisfaction problem for payouts.

\keywords{blockchain  \and ledger \and smart contract \and formal verification \and
specification \and transition system \and UTxO \and message-passing}
\end{abstract}

%such as Ethereum \cite{ethereum}, Tezos \cite{tezos}, and Zilliqa \cite{zilliqa}

\section{Introduction}
\label{sec:intro}

Message-passing is the standard for communicating data and assets between contracts in
account-based ledgers \cite{ethereum} \cite{zilliqa} \cite{tezos}.
Both sending and receiving a message is performed atomically as part of processing a
transaction containing that message. In some cases, the transaction itself
may be referred to as a message \cite{tezos}.
Communication between scripts in UTxO-based ledgers follows a different architecture.
Smart contract enabled UTxO models such as Cardano \cite{alonzo}, Algorand \cite{algorand-docs},
and Ergo \cite{ergo} use scripts only to \emph{check} that certain changes being
made by a transaction to the ledger state are allowed. A
script may require that another script executes successfully given
certain arguments, and within the same transaction. Script interaction
and communication is implemented via such constraints on other scripts, which we refer to
as \emph{dependencies}.

In this work we present an EUTxO layer-2 implementation of \emph{asynchronous message-passing} as an
alternative method of communication between individual scripts
as well as the stateful contracts they implement (Section \ref{sec:messages}).
We build this infrastructure on top of the extended UTxO ledger presented in
\cite{eutxo}, and formalized in terms of the small-step semantics formalism \cite{steps}
also used to specify the Cardano EUTxO-based ledger \cite{alonzo}. The
stateful message-passing contract is an instantiation of the structured contract
framework \cite{structured}. This framework includes the semantics for specifying
and implementing a stateful contract, together with a proof obligation that
the specification meets the implementation, formalized in Agda \cite{agda-structured}.

The state of the message-passing contract is a set of messages. This state is represented
on the ledger as a collection special NFT tokens storing message data,
which are distributed across separate UTxO entries. Messages specify
the output that was spent to validate their production. Thus, they serve as proof
artefacts of the validation of specific output-locking contracts, with specific
inputs. This gives messages a verified \emph{sender}. Similarly,
message tokens can only be spent and burned by an authenticated \emph{recipient},
another contract that must validate with particular inputs when consuming a message.
Any script is able to interface
with the message-passing contract so long as the user input to the script can be
decoded as a list of messages being produced and consumed.

The message-passing structured contract we propose addresses
some of the challenges of programming on an EUTxO ledger. It does so by
defining a way to record authentically-sourced data and assets on an EUTxO ledger,
for consumption by a specific recipient, in a highly concurrent way.
In addition to providing a way to interpret the notion of message-passing
on an EUTxO ledger, our distributed stateful contract design demonstrates
the flexibility of the structured contract formalism for implementation
of contracts with state contained in multiple UTxO entries. Unlike a centralized
design with a single UTxO entry keeping track of all messages, our design allows
arbitrary reordering of message-passing transactions whenever one of them does not
consume messages (or other outputs) produced by the other.
We demonstrate that this scheme can be used in the following ways, addressing the
corresponding challenges of EUTxO programming :

\begin{itemize}
  \item[(i)] as a authenticated output of a function call with specific
  inputs to the function, similar to the technique called memoization (Section \ref{sec:partial}) ;

  \item[(ii)] as intermediate storage of assets and data for the purpose
  of asynchronous communication between stateful contracts
   (Section \ref{sec:async});
\end{itemize}

Next (Section \ref{sec:doublesat}), we present a formalization of an as-of-yet unformalized, but widely discussed
problem in EUTxO programming :
the \emph{double satisfaction problem} (DSP) \cite{marlowe-audit-tech} \cite{plutus-docs} \cite{ds-blog}
\cite{marloweaudit}. DSP may occur when multiple scripts within
the same transaction share the same constraint, which is satisfied. For example,
by making a \emph{single} payout to an address as required
by two separate contracts, even though the intent of each of the contracts was that
the address should have received a separate payment, for a total of two.
Using the structured contract formalism, we formalize when a constraint
of a contract is vulnerable to such a situation. We then demonstrate that making
payouts via messages is not susceptible to double satisfaction.

The main contributions of this work are :

\begin{itemize}
  \item[(i)] the specification and implementation of a message-passing structured
  contract ;
  \item[(ii)] two applications of the message-passing contract : messages as
  outputs of function calls, and messages-passing as a means of asynchronous
  communication of data and transfer of assets between contracts and within a contract ;
  \item[(iii)] formalization of the double satisfaction problem ;
  \item[(iv)] a result stating that if payouts are specified via changes in the
  state of the message-passing contract, the specification of the required
  payouts is not susceptible to DS
\end{itemize}

% separate computation and communication



% This framework provides a method for
% demonstrating the correctness of the implementation of a stateful program specification.
% Specific design patterns describing a more concrete
% approach to modeling stateful programs on a EUTxO ledger, such as constraint
% emitting machines \cite{eutxoma}, correspond to subsets of the structured
% contracts framework.Our message-passing structured contract,
% supports an asynchronous exchange of messages
%
% specify an asynchronous message-passing structured contract.
% The structured contract framework \cite{structured} is a framework
% for specifying stateful programs running on the EUTxO ledger using
% the same small-steps semantics approach used for specifying ledger
% functionality \cite{alonzo}. This framework provides a method for
% verifying the correctness of the implementation of a stateful program specification.
% Specific design patterns describing a more concrete
% approach to modeling stateful programs on a EUTxO ledger, such as constraint
% emitting machines \cite{eutxoma}, correspond to subsets of the structured
% contracts framework.
%
% Stateful contracts in the extended UTxO model come with a unique
% set of programming challenges. To address some of those challenges,
% in this work we present a structured contract
% formalizing the concept of \emph{message-passing} in the EUTxO ledger.
% It constitutes a new paradigm for managing and implementing asynchronous
% communication between scripts and between structured contracts, as well as between
% scripts implementing a single structured contract.

\subsection{The \EUTXO ledger model}
\label{sec:sts}

First, we give an overview of the semantics we use for
our contract and ledger specifications,
introduced in prior work~\cite{eutxo,eutxoma,structured},
but included here for the sake of self-containment.

\bpar{Ledger types}
For the purposes of self-containment, we include a description of EUTxO ledger
model types.
We note and justify the (minimal) changes we introduce to the existing model in the description.
The types of booleans, natural numbers, and integers are denoted by
$\B$, $\N$, and $\Z$, respectively. The type $\Ix \leteq \N$ is used for indexing,
e.g. of elements in a list. The type $\Slot \leteq \N$ is used to indicate
blockchain time.

The ledger state consists of a UTxO set,
which is a collection of unspent outputs, each associated with a unique identifier.
An output is a triple
$(a, v, d) \in \Output$, where $a \in \Script$ is the address of the output,
$v \in \Value$ is the collection of assets at this address, and $d \in \Datum$ is
data specified by the user at the time of constructing the transaction which
creates this output (we give more details on these three types below).
The type of
the UTxO set is $\UTxO \leteq \OutputRef \mapsto \Output$, which is
a finite key-value map with unique keys of type $\OutputRef$. We denote a single
element in a finite key-value map $u$ (such as the UTxO set) by $i \mapsto o \in u$.
An output reference $(\var{tx}, \var{ix}) \in \OutputRef \leteq \Tx \times \Ix$ pointing
to an output $o$ in a UTxO set consists of the transaction $\var{tx}$, which
added $(\var{tx}, \var{ix}) \mapsto o$ to the UTxO set, and the index $\var{ix}$,
which is the position of output $o$ in the list of outputs of $\var{tx}$.

The type $\Value$ is used to represent bundles of multiple kinds of assets
Each type of asset in the bundle $v \in \Value$ has a
unique asset ID, $a \in \AssetID \leteq \Policy \times \TokenName$, which
identifies a class of fungible tokens. Associated to the asset ID of each
type of asset in a bundle is a a quantity $q \in \Quantity \leteq \Z$, specifying
the amount of the asset with the given ID in $v$. When $v$ contains $0$ of a
given asset type, its asset ID is not included in $v$.
An asset bundle containing one kind of asset
with asset ID $(\fun{policy},~\fun{tokenName})$
of quantity one is denoted by $\{~\fun{policy} ~\mapsto~\{ \fun{tokenName} \mapsto 1 \}~\}$.

An asset with ID $(p, t)$ has the minting (and burning) policy
$p \in \Policy \leteq \Script$. When an asset under this policy is minted or
burned, the policy script is executed to determine whether the transaction
is allowed to perform this action. The token name $t$ is specified by the user
at the time of constructing the minting transaction. It is used to differentiate
between assets under the same policy. Unlike previous work~\cite{eutxoma,structured},
where the token name is a string, we take $\TokenName \leteq \Data$.
$\Value$ forms a group under addition ($+$) with a zero element ($0$)
and a partial order ($\leq$)~\cite{utxoma}.

A script $s \in \Script$ is a piece of user-defined code that is executed as
part of transaction validation, applied to specific inputs.
Script code is stateless and produces a boolean output.
Scripts are executed as part transaction validation to check
that a transaction is permitted to do the action with which the script is associated.
Scripts are used to specify permissions for two kinds of actions: spending an output (these
are referred to as "validators", or sometimes the "address" of the output), and
minting or burning tokens (these are called minting policies).

We denote script application by
$\applyMPScript{\_}$, followed by the script arguments.
At the time of evaluation,
the arguments supplied to a script consist of transaction data (of the transaction
executing it), as well as the data about the specific action for which
the script specifies permission (i.e. the output being spent, or the tokens
being minted under the policy). An extra piece of data $d \in \Redeemer$, associated
with the particular action being validated, is specified by the user
at the time transaction construction.

On-chain data of variable type, including $\Datum$, $\Redeemer$, and $\TokenName$,
are all type synonyms for $\Data$;
for the sake of brevity,
we will omit explicit calls to the corresponding encoding/decoding functions
as these will be obvious from the types involved,
so any time a value is used as $\Data$ presupposes that decoding is successful.

Updates to the ledger state are specified in the form of a $\Tx$ (transaction) data structure.
A transaction $\var{tx} \in \Tx$
contains (i) a set of \emph{inputs}, each containing an output reference, an output,
and the associated redeemer, (ii) a list of outputs, which get entered into the
UTxO set with the appropriately generated output references, (iii) a pair of slot
numbers representing the validity interval of the transaction, (iv) a $\Value$ being
minted by the transaction, (v) a redeemer for each
of the minting policies being executed, and (vi) the set of (public) keys that signed the
transaction, alongside their signatures.

\bpar{Small-step specifications}
We formulate the transitions of ledgers and contracts
in the form of small-step operational semantics~\cite{plotkin},
as exemplified by the official specification of the Cardano ledger~\cite{shelley,alonzo}.
In our specifications and contract implementation pseudocode,
we follow standard set-theory notation, and clarify any
non-standard notation usage alongside it.

A transition relation $\mathsf{TRANS} \subseteq (\Env \times \State \times \Input \times \State)$
is a collection of 4-tuples. A member $(\var{env},~s,~\var{i},~s')$
of this relation is also denoted by :
  \begin{equation*}
    \var{env} \vdash
    \var{s} \trans{trans}{i} \var{s'}
  \end{equation*}

The variable $\var{env} \in \Env$ is the environment
of the state transition, $s \in \State$ is the starting state, $i \in \Input$ is the input,
and $s' \in \State$ is the end state.
The system $\TRANS$ is a labelled transition system.
For a given transition $(\var{env},~s,~i,~s') \in \TRANS$, the pair $(\var{env},~i)$
of an environment and an input make up the \emph{label} of this transition from $s$ to $s'$.
Conventionally~\cite{shelley}, $\var{env}$ is block-level data, such as blockchain time, whereas
$i$ is specified by the user, e.g. a transaction.

\bpar{Ledger transition semantics}
\label{sec:ledgersem}
The ledger semantics on top of which we build the results of this paper
are found in existing work~\cite{eutxo,eutxoma,structured}, but we include them here
for self-containment and in order to introduce appropriate notation.
The ledger transition system is given by the subset
$\mathsf{LEDGER} \subseteq \Slot \times \UTxO \times \Tx \times \UTxO$.
Membership in this subset is specified by a single transition rule $\fun{ApplyTx}$,
which ensures that
$(\var{slot},~\var{utxo},~\var{tx},~\var{utxo'}) \in \mathsf{LEDGER}$
whenever $\fun{checkTx}~(\var{slot},~\var{utxo},~\var{tx}) = \true$, and $\var{utxo'}$
is given by $(\{~i\mapsto o \in \var{utxo} ~\mid~ i \notin tx.\fun{outputRefs}~\}) \cup ~tx.\fun{outputs} $.
Here, the notation $\fun{r}.f$ represents accessing a (named) field $f$ of a record $\fun{r}$.
This is expressed in rule \textsc{ApplyTx} below,
where any unbound variables are implicitly considered as universally quantified.
\begin{equation*}
\inference[\textsc{ApplyTx}]
{
\var{utxo'}~\leteq~(\{~i\mapsto o \in \var{utxo} ~\mid~ i \notin tx.\fun{outputRefs}~\}) \cup ~tx.\fun{outputs}\\ ~ \\
\fun{checkTx}~(\var{slot},~\var{utxo},~\var{tx})
\\ ~ \\
}
{
\begin{array}{l}
    \var{slot}\\
\end{array}
    \vdash
    \left(
    \begin{array}{r}
    \var{utxo} \\
    \end{array}
    \right)
    \trans{ledger}{tx}
    \left(
    \begin{array}{r}
    \varUpdate{\var{utxo'}}  \\
    \end{array}
    \right) \\
}
\end{equation*}

The function $\fun{checkTx} : \Slot \times \UTxO \times \Tx \to \B$
checks the predicates which are consistent
with the EUTxO model on which this work builds~\cite{eutxoma,structured}.
This includes executing all required validator and minting policy scripts with
the appropriate inputs.
%
The projection $tx.\fun{outputRefs}$ returns a UTxO set containing an entry $k \mapsto o$
for each input $i$ of $\var{tx}$, where the key of
the entry is the output reference $k$ of $i$, and its value
is the output $o$ of $i$.
The value $\var{utxo'}$ is calculated by removing the UTxO entries in $\var{utxo}$
corresponding to those in $tx.\fun{outputRefs}$, and adding the entries
constructed by $tx$, see~\cite{structured} for details. 

\subsection{Structured contracts}
\label{sec:struc}

The structured contract framework~\cite{structured} is a formalism for specifying and demonstrating
the integrity of the implementation of a stateful contract on $\LEDGER$.
We give the definition here for self-containment and in order to introduce the
appropriate notation.
A structured contract includes a small-steps semantics specification, as well as
a ledger representation of its state and input. The ledger representation
is a pair of functions: one which computes the contract state from
a given UTxO state (or fails), and another which computes the input to the contract
for a given transaction.

For a given valid $\LEDGER$ step, the representation functions must compute a
valid step in the structured contract specification given that the starting
UTxO state corresponds to a contract state. This integrity constraint is
expressed as a proof obligation for the instantiation of a structured contract.
This design guarantees that no invalid contract state updates are ever possible
on the ledger.

Suppose $\STRUC \subseteq (\emptytypeT \times \State \times \Input \times \State)$
is a small-step transition system.
Let $\pi : \UTxO \to \State \cup \emptytypeT$ and $\pi_{\Tx} : \Tx \to \Input$
be functions such that :
\begin{equation*}
\inference
{
    \\~\\
    \pi~u~ \neq~ \emptytype &
    {
    \begin{array}{c}
        e\\
    \end{array}
    }
    \vdash
    {
    \left(
        \begin{array}{r}
        \var{u} \\
        \end{array}
    \right)
    }
    \trans{ledger}{\var{t}}
    {
    \left(
        \begin{array}{r}
        \var{u'} \\
        \end{array}
    \right)
    }
    \\~\\
}
{
    (\pi~u'~ \neq~ \emptytype) ~~\wedge~~
    \begin{array}{r}
    \emptytype \\
    \end{array}
\vdash
    (\pi~u)
    \trans{struc}{\pi_{\Tx}~t}
    (\varUpdate{\pi~u'})
}
\end{equation*}

The triple $(\STRUC, \pi, \pi_{\Tx})$ is called a \emph{structured
contract}, and we denote it by $(\STRUC, \pi, \pi_{\Tx}) \succeq \LEDGER$.
Note that $\pi$ is function with an output that is a \emph{maybe} type, 
$\State \cup \emptytypeT$, where $\emptytypeT$ is a
singleton. When $\pi~u = \emptytype$, there is no
contract state corresponding to the ledger state $u$.
The block-level data is never exposed to user-defined scripts in this model,
so that the context of a structured contract is necessarily $\emptytype \in \emptytypeT$.

\section{Message-passing in \EUTXO}
\label{sec:messages}

Conceptually, a message is data sent from a sender to a recipient~\cite{distributed}.
In our design, a message is a data structure of type $\Msg$ encoded on
the ledger in a specific way. It also includes a sender, receiver, and some data or assets.
The content of $m \in \Msg$ is encoded as the $\TokenName$ of an
NFT with the minting policy $\msgsTT$. It is encoded as such in order to
maintain certain guarantees about the message's integrity, which are ensured
by the NFT minting policy.
A message $m \in \Msg$ consists of the following fields :

\begin{itemize}
  \item[(i)] an output reference $\fun{inUTxO} : \OutputRef$. An output with
  this reference must be spent when the message token is minted;
  \item[(ii)] an index $\fun{msgIx} : \Ix$. It is used to uniquely identify a message
  whenever multiple messages are produced in association with spending a single UTxO entry;
  \item[(iii)] an output $\fun{msgTo} : \Output$. It is an output that must be spent
  to validate the consumption of the message \emph{by that recipient} (such an
  output may not be unique);
  \item[(iv)] an output $\fun{msgFrom} : \Output$. It is an output that must be spent
  to validate production of the message \emph{by that sender}. Specifically,
  the entry $m.\fun{inUTxO} \mapsto m.\fun{msgFrom}$ must be spent;
  \item[(v)] a value $\fun{msgValue} : \Value$. It specifies the assets being sent.
  When a message is minted
  and placed in an output, this output must \emph{also} contain these assets;
  \item[(vi)] data $\fun{msgData} : \Data$. It is the data being sent via this message.
\end{itemize}
%

Each message requires a unique identifier to enable some of the applications
we present later. Here, we use an approach based on the thread token mechanism
to ensure NFT uniqueness~\cite{eutxoma}.
This mechanism requires that the thread token's minting policy checks that a
particular output reference is spent from the UTxO by the minting transaction,
and exactly one token is minted under this policy.
To uniquely identify a message NFT we use the output reference $\fun{inUTxO}$,
together with the message index $\fun{msgIx}$.
Duplication of unique identifiers is forbidden by the implementing scripts.

Sending a list of messages is done by submitting a transaction that (i) \textbf{mints} the
NFTs encoding each of the messages, and (ii) for each message, spends the sender
output with a redeemer containing the list of messages "from" that output.
% Duplication is checked (and disallowed) across all lists of messages being sent in a transaction.
For an output to receive a list of messages, a transaction must spend the outputs containing the
messages, and \textbf{burn} the message tokens. It must also spend the receiver output,
and supply it with a redeemer containing the messages it is receiving.
% Up to the receiver contract code to decide what to do with the value in the
% message, and up to the sender to decide where that value comes from

The state of the message-passing contract is given by a set of messages, with
$\State_{\MSGS} \leteq \powerset{\Msg}$, which represents
messages that have been sent, but not yet received. Here, $\powerset{\Msg}$
denotes the set of all subsets of $\Msg$.
The input is the whole transaction, with $\Input_{\MSGS} \leteq \Tx$.
The $\MSGS$ transition system specifies the rules for sending and
receiving messages, see Fig.~\ref{fig:messages-rule}.

The function
\[\msgTkn ~\var{msg} \leteq \{ ~\msgsTT \mapsto \{~\var{msg} \mapsto 1~\}~\}\]
encodes a message as a message token, recording the message data as its token name,
and $\msgsTT$ as its minting policy. According to this policy, each message token minted by
a transaction must be placed into a UTxO entry locked by a special validator, $\msgsVal$,
which only checks that any message token in that UTxO entry
is burned. The message token minting policy $\msgsTT$ performs the same
checks and assignments (1, 3, 5, 6, 7) that are in the $\MSGS$ specification in Fig.~\ref{fig:messages-rule},
with the notable exception of checking the non-duplication of existing messages,
as required by (2). This cannot be checked explicitly by $\msgsTT$ because
it cannot inspect the global set of existing messages under this policy, and
must instead be proved as a consequence of the generation of the message's unique
identifier. The type of the decoded redeemer for both $\msgsTT$ and
$\msgsVal$ is $\{\emptytype\}$, as they are not used in the implementation.

The notation $[a1; ... ; ak] : [A]$ represents a list of type $A$, with concatenation
denoted by $\app$. The predicate $\_ \# \_$ takes two lists, returning $\true$ if
they are disjoint, and $[ \fun{f}~a~ \mid~ a \leftarrow \var{as}]$ denotes list comprehension.
The contracts $\msgsTT$ and $\msgsVal$ implementing the $\MSGS$ specification
are given in Fig.~\ref{fig:msgs-codeTT} and~\ref{fig:msgs-codeV}.

The projection function
$\pi_{\Msg}$ returns, for a given $utxo$, all messages encoded in
the message tokens that exist in the UTxO set. It
returns $\emptytype$ when one or more messages have been duplicated or outputs
incorrectly generated in the $utxo$. This is guaranteed by $\fun{msgOutsOK}$,
see Fig.~\ref{fig:msgs-aux} for the details.
\begin{align*}
  \pi_{\Msg}~utxo &\leteq \begin{cases}
    \{~m~\mid~\_\mapsto o \in utxo, \msgTkn~m~ \subseteq o.\val \} & \text{ if } ~\fun{msgOutsOK}~utxo \\
    \emptytype & \text{ otherwise }
  \end{cases}
\end{align*}

We give a proof sketch of the simulation relation between $\LEDGER$
and $\MSGS$ in the extended version of this paper. Recall that this relation
ensures the integrity of the implementation,
i.e. that the implementation of $\MSGS$ via the $\msgsTT$
and $\msgsVal$ scripts only allows ledger updates that are mapped to \emph{valid} $\MSGS$
transitions (by the $\pi$ and $\pi_{\Tx}$ projections).
%
\begin{figure}[htb]
  \begin{equation*}
    \inference[\textsc{Process}]
    {
    \text{\textcolor{gray}{(1) construct a list of messages encoded in redeemers}} \\
    \var{sndMsgs} \leteq [~ (\var{msg},~i)~\mid~i\leftarrow~tx.\inputs,
    ~(sr,~\var{msg})~ \leftarrow~(i.\redeemer),~sr=\send~] \\
    \var{rcvMsgs} \leteq [~ (\var{msg},~i)~\mid~i\leftarrow~tx.\inputs,
    ~(sr,~\var{msg})~\leftarrow~(i.\redeemer),~sr=\receive~]
    \\~\\
    \text{\textcolor{gray}{(2) check that no new messages are duplicates}}\\
    [~\fun{getMsgRef} ~m \mid (\_,\ m) \leftarrow \var{newOuts}~]~~\#~~
    [~\fun{getMsgRef} ~m \mid (\_,\ m) \leftarrow \var{usedInputs}~]~~\\ \#~~
    [~\fun{getMsgRef} ~m \mid m \leftarrow \var{msgs}~]
    \\~\\
    \text{\textcolor{gray}{(3) compute the set of message token-containing outputs being created}} \\
    \var{newOuts}~\leteq ~\{~(o,~\var{msg})~\mid~o~\in~tx.\outputs,~
    \msgTkn~\var{msg}~\subseteq~o.\val~\}
    \\~\\
    \text{\textcolor{gray}{(4) check that all the messages are correctly constructed : correct sender output,}}\\
    \text{\textcolor{gray}{sender has correct redeemer, output reference is spent, one message per output,}}\\
    \text{\textcolor{gray}{output containing message token has correct validator and sufficient value}}\\ %
    \forall~(o,~\var{msg})~\in~\var{newOuts},~(\var{msg},~(\var{msg}.\fun{inUTxO}, \var{msg}.\fun{msgFrom}, \_))~\in~\var{sndMsgs} \\
    \wedge~\{~t~ \subseteq~o.\val~\mid~\dom~t~=~\{\msgsTT\}~\} = \msgTkn~\var{msg} \\
    \wedge~o.\validator~=~\msgsVal ~\wedge~o.\val~\geq~\var{msg}.\fun{msgValue}
    \\~\\
    \text{\textcolor{gray}{(5) compute the set of message token-containing outputs being spent}} \\
    \var{usedInputs} \leteq \{~ (i,~\var{msg})~\mid~
    i~ \in~ tx.\inputs,~\msgTkn~\var{msg}~\subseteq~i.\fun{output}.\val~\} \\~\\
    \text{\textcolor{gray}{(6) check that all messages are correctly consumed : }}\\
    \text{\textcolor{gray}{the receiver output is correct, input has correct redeemer, and message exists}}\\
    \forall~(i,~\var{msg})~\in~\var{usedInputs},~(\var{msg},~(\_, \var{msg}.\fun{msgTo}, \_))~\in~\var{rcvMsgs}~\wedge~\var{msg} \in \var{msgs}\\
    \\~\\
    \text{\textcolor{gray}{(7) check minting and burning of message tokens : }}\\
    \Sigma_{(\_,\var{msg})~\in~\var{newOuts}}~\msgTkn~\var{msg} ~~+~~
    \Sigma_{(\_,\var{msg})~\in~\var{usedInputs}}~(-1) ~*~ (\msgTkn~\var{msg}) ~\\
    ~~~~~~~~~=~ \{~\msgsTT \mapsto tkns~\in~tx.\mint~\}
    \\~\\
    }
    {
    \begin{array}{l}
      \emptytype \\
    \end{array}
      \vdash
      \left(
      \begin{array}{r}
        \var{msgs}
      \end{array}
      \right)
      \trans{msgs}{tx}
      \left(
      \begin{array}{r}
        \varUpdate{%
(\var{msgs} \setminus [~m \mid (\_,\ m) \leftarrow \var{usedInputs}~])} \\
\varUpdate{\cup [~m \mid (\_,\ m) \leftarrow \var{newOuts}~]}
      \end{array}
      \right) \\
    }
  \end{equation*}
  \caption{Specification of the $\mathsf{MSGS}$ transition}
  \label{fig:messages-rule}
\end{figure}


\section{Message-passing use cases}
\label{sec:usecases}

In this section we discuss applications of the message-passing structured contract.

\subsection{Memoization}
\label{sec:memo}

There may be strict resource use constraints that apply to executing code on a
blockchain. It may not be possible for a transaction to run the code of a large contract in its
entirety. It may be desirable to divide such code into less memory- and CPU-intensive
functions whose outputs are pre-computed for use by an aggregate function.
A script may not trust values pre-computed off-chain,
so a proof that a value was correctly computed on-chain is required.
In this section we describe a technique for constructing such proofs using
the $\MSGS$ contract. It is
similar to a specific kind of caching called \emph{memoization}~\cite{memoization},
which is also how we refer to our approach.

Consider a function $\fun{myFunction} : \mathsf{MyInType} \to \mathsf{MyOutType}$
which performs some computation. We define a script $\mathsf{checkMyFunction}$
(Fig.~\ref{fig:checkmf}), which wraps the computation
done by $\fun{myFunction}$. This script mints a message token with
data $(\var{fIn},~\var{fOut})$, such that $\fun{myFunction}~\var{fIn}~=~\var{fOut}$,
and a script $\mathsf{useMyFunction}$ (Fig.~\ref{fig:usemf}) that can
consume a message with the redeemer $[(\receive,~m)]$
when $m$ is addressed to an output locked by $\mathsf{useMyFunction}$, and is
sent by an output locked by $\mathsf{checkMyFunction}$.
This message serves as a proof that $\fun{myFunction}~\var{fIn} = \var{fOut}$,
so, $\mathsf{useMyFunction}$ can perform a
computation $\fun{checkStuff}$ relying on the fact that $\fun{myFunction}~\var{fIn} = \var{fOut}$.
Note that $\fun{msgTo}$ is not constrained by this contract, so that the generated
message can be addressed to any recipient.

We give the result that formalizes the use of message-passing
to prove that $\fun{myFunction}~\var{fIn} = \var{fOut}$.

\paragraph{Lemma (Verified input-output pairs). }
\label{lem:io-pairs}
For any $(s,~u,~tx, u') \in \LEDGER$, with $\pi~u \neq \emptytype$ and
$(i, ~(\mathsf{useMyFunction}, v, d),~ r) \in tx.\fun{inputs}$,
such that
\begin{align*}
  [(\receive,~m)] &= r \\
  (\var{fIn},~\var{fOut}) ~&=~m.\fun{msgData} \\
  m.\fun{msgFrom} &= (\mathsf{checkMyFunction}, \_, \_)
\end{align*}
%
necessarily $\fun{myFunction}~\var{fIn} = \var{fOut}$, and
$m.\fun{msgTo} = (\mathsf{useMyFunction}, v, d)$.
For a proof sketch, see the extended version of the paper.
Note here that the memoization approach we presented can be viewed as a kind of
\emph{untrusted oracle}. The computation done to produce the memoized
input-output pair cannot be falsified, so that no trust is required to make
use of it.

\subsection{Contracts using message-passing}
\label{sec:async}

Stateful contract interaction, or communication, in the EUTxO model is implemented
via dependencies~\cite{eutxoma}. A \emph{dependency} of a script $c$ is a constraint requiring that another
script $c'$ must be executed within the same transaction, possibly with specific
arguments. Using the $\MSGS$ contract to implement communication between contracts
reduces ad-hoc reliance on arbitrary
script dependencies, and makes contract interaction more principled and
amenable to formal verification.

We say that stateful contracts \emph{use message-passing}
when they require the production or
consumption of messages to or from scripts implementing the contract.
We formalize this notion in this section. Note that due to space constraints, we
omit several interesting results about contracts using message-passing, as well as
a detailed example of a contract that makes payouts via messages, all of which can be found
in the extended version of this work.

Message-passing specification is closely integrated with ledger semantics,
and inspects the scripts, redeemers, and datums of the input transaction.
Because of this, a message-passing contract must also inspect these
in order to correctly construct a message. So, a state projection function
for a contract that uses message-passing includes the UTxO entry relevant
to the contract state, in full. The contract input is the complete transaction.

Suppose that $\fun{F} : \Output \mapsto \B$ is a constraint on outputs, and
$\fun{c} : \UTxO \to \B$ is a constraint on a valid UTxO state. The contract
denoted by $(\pi_{\fun{Fc}},~\pi_{\Tx},~\STRUC)$ is a structured contract with
\begin{align*}
  \State &\leteq \{ i \mapsto o~\in~u ~\mid ~u \in \UTxO,~\fun{F}~o~ \} \\
  \pi_{\fun{Fc}}~ u~&\leteq ~\begin{cases}
    \{ i \mapsto o ~\in u~\mid ~\fun{F}~o \} & \text{ if } \fun{c}~u \\
    \emptytype  & \text{ otherwise }
  \end{cases}~\\
  \pi_{\Tx} &\leteq \fun{id}
\end{align*}

We can combine $\STRUC$ and $\MSGS$ to construct
the structured contract $\STRUC_{\MSGS}$,
\begin{align*}
  \pi_{\State-M}~u~&\leteq~\begin{cases}
    (\pi_{\fun{Fc}}~u, \pi_{\Msg}~u) & \text{ if~~ } \pi_{\fun{Fc}}~u~\neq~\emptytype~\neq~ \pi_{\Msg}~u \\
    \emptytype & \text{ otherwise}
  \end{cases} \\
  \pi_{\Tx-M}~&\leteq~\fun{id}_{\Tx} \\
  \STRUC_{\MSGS} &\leteq \{~ (\emptytype, (s, m), tx, (s', m'))~\mid~
  (\emptytype, s, tx, s') \in \STRUC, (\emptytype, m, tx, m') \in \MSGS~\}
\end{align*}

We call this contract \emph{message-augmentation} of $\STRUC$.
We define the following functions that filter messages sent or received by $\STRUC$ :
\begin{align*}
  \fun{getFromSTRUCmsgs} ~\var{msgs} & \leteq \{~m~\mid~m~\in \var{msgs},~\fun{F}~(m.\fun{msgFrom})~\} \\
  \fun{getToSTRUCmsgs} ~\var{msgs} & \leteq \{~m~\mid~m~\in \var{msgs},~\fun{F}~(m.\fun{msgTo})~\}
\end{align*}

\paragraph{Definition (Uses message-passing). }
\label{def:usesmp}
We say that $\STRUC$ \emph{uses message-passing} whenever the set defined by
\[ \fun{getMSGS}~(\emptytype, (s, m), tx, (s' m'))~\leteq~\fun{getFromSTRUCmsgs}~(m' \setminus m)
\cup \fun{getToSTRUCmsgs}~(m \setminus m')  \]

is non-empty for some $(\emptytype, (s, m), tx, (s' m')) \in \STRUC_{\MSGS}$. \newline

We define the set of \emph{payouts} in the step $(\emptytype, (s, m), tx, (s' m')) \in \STRUC_{\MSGS}$ by
\begin{align*}
  \fun{getPayouts}~(\emptytype, (s, m), tx, (s' m'))~\leteq~\{ \var{msg} \in \fun{getFromSTRUCmsgs}~(m' \setminus m) ~\mid \\ ~~~~\var{msg}.\fun{msgValue} > 0 \wedge
  \neg~(\fun{F}~(\var{msg}.\fun{msgTo}) ~\}
\end{align*}

Whenever this set is necessarily non-empty for some step in $\STRUC_{\MSGS}$, we say that it
\emph{makes payouts with messages}.

\bpar{Discussion}
A contract is said to use message-passing whenever there is a step in $\STRUC_{\MSGS}$ that
requires the production or consumption of a non-empty set of messages to or from $\STRUC$.
Some computation performed by contracts implementing $\STRUC$ may be contingent on receiving a specific
message. For example, accepting a payment message sent by another contract.
 %The lemma \ref{lem:msgs-ok} states

Contracts that use message-passing share common features that are both necessary
and sufficient for a script $c$ implementing the contract to be able to
interface with the message-passing contract :
(i) the script's redeemer must decode to a list of sent/received messages, and (ii)
the script must ensure that the corresponding messages are included in the
transaction's mint field.

For a given step $(\emptytype, s, t, s') \in \STRUC$, we refer to the messages sent and received
by outputs that make up $s$, i.e. those filtered by $\fun{F}$, as a script's
\emph{communication}. Calculating $s'$ for the given $(s, t)$
is the $\STRUC$ contract's
computation. $\STRUC$ may still include arbitrary dependencies on scripts implementing
contracts other than $\MSGS$. Specifying when a contract has no non-message
dependencies is important for determining when it is guaranteed to be able to progress.
This is, however, the subject of future work.

\section{Messages as payouts}
\label{sec:msgs-pay}

A \emph{payout} is a message that is from $\STRUC$, but not addressed to $\STRUC$, and
specifies a sent value greater than zero. The function that
returns all the payouts for a given contract, $\fun{getPayouts}$, is a function
of the start and end $\MSGS$ states \emph{only}.
Consider a transition $(\emptytype, (s, m), tx, (s' m')) \in \STRUC_{\MSGS}$.
To guarantee that a message-payout $\var{msg}$ is made whenever rule $R \subseteq \STRUC_{\MSGS}$
applies, $R$ must ensure that (i) $\var{msg} \in \var{m'} \setminus m$,
(ii) $\fun{msgValue}~\var{msg} ~>~ 0$, and (iii) $\var{msg}$ must be \emph{from} an output
locked by some script $c$ implementing $\STRUC$.
The script $c$ implementing this constraint of $\STRUC$ should
instead include the constraint $\msgTkn~\var{msg} \in \var{tx} . \mint$. For a detailed example
of making payouts via messages, see the extended version of this work.
Making payouts in this way has an advantage over the naive approach to payouts.

\bpar{$\MSGS$ payouts and double satisfaction}
In Section \ref{sec:doublesat} we present a naive approach to payouts. This approach is
vulnerable to DS, since the constraint requiring a payout to be made is expressed
as a function the input transaction, rather than the state.
Naive payout outputs can be produced and consumed by any valid transaction at any time,
independently of the state update of any contract.
Without a mechanism to \emph{associate a payout with its sender}, is not possible to
include naive payouts in a contract's state.

Intuitively, making payouts via messages provides such a mechanism by ensuring that the
sender of the payout is recorded in the message token, and that the message token
has a unique identifier. Formally, since making payouts via messages can be expressed as
a predicate on a the start and end $\MSGS$ states, rather than on the input transaction,
constraints on message payouts are not vulnerable to DS for a message-enhanced contract.
We can express this as follows (see Appendix~\ref{sec:appendix-proofs} for the proof):

\paragraph{Lemma ($\MSGS$-payouts and DS)}
Suppose $(\pi_{\fun{Fc}},~\pi_{\Tx},~\STRUC)$ is a structured contract,
and $\STRUC_{\MSGS}$ is its message-enhanced version.
Let $\fun{C} \supset \STRUC_{\MSGS}$ be a constraint expressible in terms of
some predicate $\fun{C'}$ on the set of payout messages,
\[\fun{C}~(\emptytype, (s, m), tx, (s' m')) \leteq \fun{C'}~(\fun{getPayouts}_{\STRUC}~(\emptytype, (s, m), tx, (s', m'))) \]

Then, the contract $\STRUC_{\MSGS}$ is not vulnerable to DS with respect to $\fun{C}$.

\section{The problem of double satisfaction}
\label{sec:doublesat}

In the EUTxO model, scripts place constraints on the transactions executing them.
\emph{Multiple scripts} may place the \emph{same constraint} on the data
of a given transaction. The issue with certain undesirable instances of this situation
is called the \emph{double satisfaction problem} (DSP).
The DSP has been discussed in Plutus documentation,\footnote{%
\url{https://plutus.readthedocs.io/en/latest/reference/writing-scripts/common-weaknesses/double-satisfaction.html}
} and in the context of contract audits,\footnote{%
\url{https://medium.com/@vacuumlabs_auditing/cardano-vulnerabilities-1-double-satisfaction-219f1bc9665e}
}\footnote{%
\url{https://github.com/tweag/tweag-audit-reports/blob/main/Marlowe-2023-03.pdf}
}
but has not yet been formally analyzed.

The DSP applies to scripts and structured contracts
that require transactions to make payouts, so it is frequently
encountered in EUTxO script programming. A \emph{naive} payout is a constraint of the
form "the transaction must include an output containing value $v$, with address $a$".
When two structured contract implementations both place such a constraint on a transaction,
it may be satisfied by a single transaction output $(a, v, \_)$, resulting in
insufficient payment made to $a$. Note here that we use
the notation $\_$ to represent a term whose value is not
relevant to the computation in which it appears.

To formalize this kind of vulnerability, let us assume that all systems discussed
in this section are deterministic (i.e. have exactly one end state for each
pair of input and start state),
and define the following function, which returns all pairs of states in all valid transitions
of a given structured contract $\STRUC$ :
\[ \fun{s}~\STRUC = \{~ (s, s')~\mid~\exists~i,~(\emptytype, s, i, s') \in \STRUC~\} \]

\paragraph{Definition (transition constraint). }
A \emph{constraint} of a transition system $\STRUC$ is a subset
$\fun{C} \subseteq \emptytypeT \times \State \times \Input \times \State$
such that $\STRUC \subseteq \fun{C}$.

\paragraph{Definition (double satisfaction). }
\label{def:ds}
A structured contract $(\pi, \pi_{\Tx}, \STRUC)$ is \emph{vulnerable to double satisfaction}
with respect to a constraint $\fun{C}$
whenever there exists another contract $(\pi, \pi_{\Tx}, \STRUC')$, with $\STRUC \subseteq \STRUC'$
and $\fun{s}~\STRUC = \fun{s}~\STRUC'$,
such that $\STRUC' \cap \fun{C} \subsetneq \STRUC'$.

\paragraph{Example ($\TOGGLE$ with extra constraint). }
\label{ex:toggle}
Consider a $(\pi, \pi_{\Tx}, \TOGGLE)$ contract, with $\State \leteq \B$ (i.e. Boolean),
and $\Input \leteq (\toggle \cup \{\emptytype\}) \times \Interval{\Tick}$. \newline


\begin{minipage}{.45\textwidth}
  \begin{equation*}
    \inference[\textsc{NoOp}]
    { \\
    }
    {
    \begin{array}{l}
      \\
    \end{array}
      \vdash
      \left(
      \begin{array}{r}
        x
      \end{array}
      \right)
      \trans{TOGGLE}{(\emptytype, \_)}
      \left(
      \begin{array}{r}
        x
      \end{array}
      \right) \\
    }
  \end{equation*}
\end{minipage}
\begin{minipage}{.45\textwidth}
  \begin{equation*}
    \label{eq:j}
    \inference[\textsc{Toggle}]
    {~\\
    5 \leq j < k \leq 9 \\~\\
    }
    {
    \begin{array}{l}
      \\
    \end{array}
      \vdash
      \left(
      \begin{array}{r}
        x
      \end{array}
      \right)
      \trans{Toggle}{(toggle, [j, k])}
      \varUpdate{\left(
      \begin{array}{r}
        \neg~x
      \end{array}
      \right) } \\
    } \\
  \end{equation*}
\end{minipage}

We define a contract $\STRUC'$ by removing $5 \leq j < k \leq 9$ from rule \textsc{Toggle},
assume it has the same projections $\pi, \pi_{\Tx}$ as $\TOGGLE$,
and define the constraint
\[ \fun{C} (\emptytype, \_, (t, [j,k]), \_) \leteq (t = \toggle) \Rightarrow (5 \leq j < k \leq 9) \]

The contracts $\STRUC'$ and $\STRUC$ transition between the same states:
$\fun{s}~\STRUC = \fun{s}~\STRUC' = {x \mapsto x,\ x \mapsto ~\neg~x}$.
Note that $\STRUC = \STRUC' \cap \fun{C} \subsetneq \STRUC'$, hence
$\STRUC$ is vulnerable to DS with respect to $\fun{C}$.

\bpar{Discussion}
Vulnerability to the DSP comes from the lack of association of some property of
transaction data with a \emph{specific structured
contract} (or script implementing it) that requires this property to hold.
Our DS definition formalizes the association between a transaction property and a
contract by defining a property to be associated with a contract only in the case
when the a property can be expressed as a predicate on the contract state update. Thus,
any property not expressible as a property of a contract state update is vulnerable
to DS.

This defines a broad class of constraints
that are vulnerable to DS, but for which this vulnerability may not necessarily be problematic.
The preceeding example demonstrates an unproblematic constraint, $\fun{C}$, which only requires
that a toggle action happens at a particular time interval. It is possible that multiple
contracts require that some other action be performed in that same time interval
simultaneously.

The onus is on the structured contract author to determine for which constraints placed on the
transaction by the contract double satisfaction vulnerability is a problem.
Then, changes must be made to the contract to ensure that the vulnerability is
removed. A general approach to mitigating arbitrary instances of DSP vulnerability
is outside the scope of this work, however,
in Section~\ref{sec:msgs-pay}, we propose a solution to the DSP for payouts.
A similar approach can be taken for DSP vulnerability for pay-ins.
It is possible to define classes of contracts that are never vulnerable to DS.
See the extended version of this work for a proof of the following lemma:

\paragraph{Lemma (DS-free contracts). }

A deterministic structured contract $(\pi, \pi_{\Tx}, \STRUC)$
is not vulnerable to double satisfaction with respect to any
constraint whenever for any $(s, i)$, there exists
an $s'$ such that $(\emptytype, s, i, s') \in \STRUC$. \newline

\bpar{DSP mitigation}
An existing heuristic for addressing the DSP
is to include a constraint in the implementing script(s)
that forces them to fail if \emph{any other scripts} are being run by the transaction.
This effectively mitigates negative consequences of potential vulnerabilities
of the given contract's constraints to the DSP. This is likely not a practical solution in many
cases, however, as it is too restrictive.
We note that, like the above example, this constraint is not on the
contract state update, but rather,
on the transaction. This means that it is itself vulnerable to DS. However,
vulnerability of this constraint to DS will likely not be deemed to be a problem
by script authors, as the purpose of introducing it is to mitigate the negative consequences of
other constraints' vulnerabilities.

\section{Discussion}
\label{sec:discussion}

\subsection{Related work}
\label{sec:related}

In this work we presented a stateful contract implementing asynchronous message-passing
in the EUTxO ledger. Message-passing is the backbone of distributed computing \cite{distributed}
\cite{design}. The $\pi$-calculus process calculus has been developed to formalize
processes and message-passing between them in a distributed computing scenario
\cite{picalc}. We conjecture that it may be possible to apply this formalism to
message-passing between structured contracts, considering them as processes
under certain circumstances. However, this is the subject of future work.

The UTxO ledger design (first introduced as the BitCoin ledger \cite{bitcoin}),
and EUTxO ledger implementations \cite{algorand} \cite{ergo} \cite{nervos} are themselves
message-passing schemes, wherein a transaction is a message to a script.
Our scheme reinterprets messages in a way that allows them to have a single verified
sender output, and a receiver that is also an output. The content of the message
is constrained as well. The contract $\MSGS$
can be viewed as a kind of linear sub-ledger within $\LEDGER$, one that may be
used as a tool in specification and verification of properties, such as temporal
properties of communicating contracts, or ones needed for the applications we presented, e.g.
Lemmas \ref{lem:io-pairs} \ref{lem:msgs-ok}.

Some account-based ledger designs \cite{ethereum} \cite{tezos} \cite{zilliqa}
use on message-passing as the default way of communication between contracts.
However, in all three cases, the message-passing is synchronous, so that both
the sender and receiver accounts are updated as part of processing a single transaction.
Our goal of separating communication from computation for stateful contracts
on the EUTxO ledger is inspired by the Scilla \cite{scilla} programming language.
Even though it was developed for the account-based ledger model, the communicating
automata structure it uses to model contracts could potentially be used to
describe message-passing structured contracts as well.

CoSplit, presented in \cite{sharding}, is a static analysis tool for implementing
\emph{sharding} in an account-based ledger model. Sharding is the practice of
separating contract state into smaller
fragments that can be affected by commuting operations, usually for the purposes of
increasing parallelism and scalability. Our message-passing contract is
distributed, so that each message is contained in a distinct output.
There is no specially-marked output containing the
consolidated message state, whose output reference a transaction must
include every time \emph{any} message is produced or consumed.
So, our design does not put superfluous constraints on the order in which multiple
transactions producing and consuming
messages can be applied to the EUTxO ledger. It does not require further state sharding.

A version of asynchronous message-passing is implemented in the ERC-20 Ethereum contract for
fungible tokens \cite{ethereum-docs}. To transfer an amount of tokens from a sender to a receiver,
the total amount being transferred must first be sent and recorded in an intermediate
data structure, then received and withdrawn from the data structure. The total
amount does not have to be withdrawn in its entirety, which is different from
our design, where a message can only be consumed in full. The ERC-20 design is
also primarily for asset transfers, whereas ours can be used to authenticate
script computation outputs as well. We also note that in an
account-based ledger, transactions interacting with the same stateful
contract like ERC-20 can usually be reordered. However, implementing message-passing
via a centralized data-storage contract in an EUTxO ledger would significantly
reduce the possibility of reordering message-passing transactions.

Formalization of blockchain and ledger functionality forms a foundation for
rigorous reasoning about smart contracts security, discussed in the
detailed overview \cite{leveragingfm}. Mathematical models of
EUTxO and UTxO ledgers and smart contracts therein, including ours, often specify
a simplified version of actual implementations
\cite{eutxoma} \cite{blockalg} \cite{nester} \cite{algorandformal} \cite{bitcoinformal}.
Languages for writing verifiable contracts such as \cite{bitml} \cite{scilla}
make programming on a blockchain ledger more amenable to formal verification.
In this work we continue this tradition of formalizing ledgers, smart contracts, and communication
models between ledger contracts, and formally studying their vulnerabilities. In particular,
this work builds on the work
done in \cite{structured} \cite{eutxoma}
on formalizing stateful program models and their security properties.

\section{Future work}
\label{sec:future}

The scheme we presented in Figure \ref{fig:messages-rule} is such that
the outputs that must be spent in order to
consume a given message are fully specified
(via the $\fun{msgTo}$ field of the message), including their
scripts, values, and datums. In future work, this constraint
could be relaxed for a more permissive system design. For example, the $\fun{msgTo}$
field can be replaced by a constraint on some part of the output. For example,
the recipient output must contain a particular NFT token in its value.
Making it an option to not even specify the receiver allows the message-passing
system to also function as a kind of broadcast system for authenticated data.
Messages broadcast in this way can serve as untrusted oracles.

A \emph{time of expiry} can be added to the message structure and used to specify a time
after which a message can be consumed under constraints other than the
spending of the intended recipient output. This would allow structured contracts
to retract any assets sent via a message but not received by a set time.
Changing the type of the message-passing redeemer from a list of messages
to \emph{either} a list of messages \emph{or} some other data can also
make $\MSGS$ more versatile. It may be useful in
enabling a given script to participate in either computation or communication
as a result of applying a transaction, depending on the redeemer specified.

In this work, we did not specify trace-based properties of $\LEDGER$
or any structured contract $\STRUC$. This topic, in general, is the subject of future
work. Of particular interest are structured contracts that can be guaranteed
to take a step without the need for executing "external" scripts, i.e. ones that
specify permissions for updating parts of the ledger which not reflected in the
contract state representation. It is often not realistic for a contract to
successfully take a step without \emph{any} external contracts validating.
For example, paying into a contract is likely to require an "external" output to be
spent within the same transaction in order to cover the payment. However, it
seems feasible to limit a structured contract's dependencies to message-passing only.
Formalizing and proving properties about this subclass of contracts in the future is
of interest.

The formalization of the double satisfaction problem we presented does not address
the possibility that a script may contain a constraint preventing any other
scripts from executing as part of carrying transaction validation. While this
seems like an obvious way to prevent the problem, we note that "no other
scripts can run during validation of this transaction" is a constraint
on the transaction that is in fact itself vulnerable to double satisfaction.
It is incidental that no other scripts that may benefit from the satisfaction of
this constraint.
This is a special case we can address in the future, however, as mentioned
earlier, it is not likely that a sensible contract can be guaranteed to progress
under the condition that the transaction runs no "external" scripts.

In the future, we intend to mechanize this contract and its applications in Agda,
using the EUTxO ledger and structured contracts model \cite{agda-structured},
as well as in the more realistic Agda-mechanized ledger model \cite{agdaspec} of
the Cardano ledger \cite{cardano}, which is currently in development.

\subsection{Conclusion}
\label{sec:conclusion}

As discussed above, some security properties of
stateful smart contracts in the EUTxO
ledger model have been formalized in existing work.
However, in such models, communication between contracts was not a focus, and was
incidentally treated the same as any other script dependency. In this work, we focus on
formalizing communication of data and assets among
scripts as well as the stateful contracts they implement. We encode communication
as a special kind of dependency of a script (or structured contract) on a distributed
stateful
contract keeping track of all unconsumed messages. Moreover, the contract
we present is specified in the
same small-steps semantic framework as the ledger itself. This allows us to reason about
it separately from other dependencies and from script computation, making it more
amenable to formal reasoning.

As an example of such reasoning, we present
two usecases. The first usecase is a variation on memoization,
wherein message tokens serve as proof
artefacts of successful script computations. The second usecase formalizes
the idea of a script communicating via message-passing, and defines when a message
constitutes a "payout". We go on to formalize the pervasive double satisfaction
problem as a property of a labelled transition system, and demonstrate
the use of message-passing to address this vulnerability in certain cases.



% specifying and reasoning about behaviour of
% stateful programs running on a EUTxO ledger, which we call the structured contract
% formalism. Our formalism defines a robust
% way to relate stateless predicate scripts executed at the ledger level to the
% corresponding corresponding executions of a specific stateful program.
% We used the well-established concept of a subsystem
% to define this relation, and a small-step semantics style already in use in
% existing systems (i.e. Cardano) for the specification of the ledger and contracts.
% This work presents a broadly applicable and principled
% way of reasoning about stateful programs on the EUTxO ledger.
%
% This paper lays the groundwork for treating ledger-implemented programs as
% formal subsystems of the ledger. This is done on a level that is very specific
% to the details of the EUTxO ledger. In the future, we aim to generalize our
% research to be applicable to other kinds of ledger transition systems.
% We would also like to better align our findings with existing concepts in the
% theory of simulation, concurrency, and distributed computation, in order to
% apply the full gamut of results in those areas for studying stateful
% programs on the ledger.
%
% Another goal for future work is to
% find a way to use the findings presented here to verify existing contracts.
% This may be done by first constructing the ledger and transaction representation
% projection functions, then building a state transition specification induced by
% $\LEDGER$ for those projections. Properties of the resulting system can then
% be studied.
%
% A full mechanization in Agda of the results and definitions in this work is currently
% under way, as it is a natural next step. We also intend to build a mechanization of this work
% integrated with the Agda mechanization of a more sophisticated and realistic
% ledger, the Cardano ledger. The Cardano
% ledger is in a unique position to be amenable to the structured contract framework
% approach to verification due to the existence of a mechanized small-steps specification
% in Agda of the entire ledger (currently in development) \cite{agdaspec}.
% Agda automation of proof generation for the subsystem proof obligations, such
% as using an SMT-solver, will be a natural progression of the project.
%
%
% Code errors and design flaws have been very costly for users
% since the introduction of smart contracts on the Ethereum platform \cite{survey}.
% Among the most high-profile and costly being the DAO hack \cite{DAO},
% and more recently, a faulty NFT contract \cite{nftbug}. Formal methods are being
% used to find, prevent, and mitigate vulnerabilities on different ledger models, and via different
% approaches \cite{formal}.
%
%
%
% Scilla \cite{scilla} is a intermediate-level language for writing smart contracts as state
% machines on an account-based ledger model.
% The Scilla authors have used Coq to reason about contracts written in Scilla, proving a variety of
% temporal properties such as safety, liveness, and others. The goal of this work,
% however, not to study stateful program behaviour, but rather, it is to formalize
% the notion of correct implementations
% of stateful programs on a platform where programs are inherently stateless. The purpose of the
% properties we discuss is to exemplify how reasoning about a
% specification trace guarantees the conclusions to hold for any
% ledger representation inducing a correct its implementation. A full treatment
% of lifting safety and liveness properties from specification to implementation
% is the subject of future work. Additionally, we believe that the subsystem approach
% we present in this work may be used to model on-chain interactions between
% Scilla contracts.
%
% The Bitcoin Modelling Language (BitML) \cite{bitml} allows the definition of smart
% contracts running on Bitcoin by means of a restricted class of state machines.
% The BitML state machines are less expressive than the class of specifications
% considered in our model, since we assume that our stateless scripts are written
% in a Turing-complete language. However, the goal of this language is similar to
% ours - to guarantee that the behaviour of certain state machines (in the case of BitML, ones defined
% using this language) if in accordance with the changes made by valid transactions, i.e. soundness.
% The difference is that we present a framework in which one can define a state
% transition system with an implementation (both using a Turing-complete language),
% then prove soundness to achieve a "correct" implementation, whereas BitML allows users to
% define a sound state machine from smaller class, then compile it to a \emph{specific}
% implementation. With BitML, LTL formulas can be automatically
% verified using a dedicated model checker. In the future, we plan to add support for LTL formulas in
% our framework.
%
% VeriSolid \cite{verisolid} synthesises Solidity smart contracts from a state machine specification, and
% verifies temporal properties of the state machine using CTL. The underlying
% ledger model for VeriSolid is, however, an account-based model, rather than the
% EUTxO model we work with. Moreover, in contrast
% to the VeriSolid approach, our approach
% relies on the contract author to themselves to synthesize an implementation
% that meets the requirements specific to the contract being built, and then
% provides a proof obligation to show that implementation is correct. This allows
% for more flexibility in the implementation, as well as in the logic
% used in checking properties. Here, again, it may be of interest to use an approach
% similar to the structured contracts framework to model interactions between
% structured contracts.
%
%
% Our work allows users to compose contracts whose state is distributed across multiple
% UTxOs and tokens on the ledger, and provides a way to formally guarantee that the update of the
% full aggregated state is in accordance with the update of its ledger representation.
% This application of structured contracts serves a similar purpose as for the EUTxO
% ledger as sharding does for an account-based one, as it can be used to increase
% parallelism and scalability. We note that the UTxO model is a natural fit for
% such state separation, since one of the benefits of such a ledger is that
% all operations either commute or fail \cite{parallelism}. Therefore, any
% UTxO state representation,
% with any (correct) implementation, will afford the relevant properties for a given
% distributed contract.
%
% On the EUTxO ledger, the constraint-emitting machine design pattern \cite{eutxoma} makes
% a formal correctness guarantee
% similar to the proof obligation we require as part of the definition of smart contracts.
% However, it is limited to a ledger representation of contract state that is
% strictly the datum and value of a
% single UTxO entry, expressing dependencies on other scripts via a limited set of
% possible constraints on
% transactions. Our model allows the contract state to be computed from multiple
% UTxO entries and tokens aggregated across the ledger state, with its evolution
% coordinated by multiple different scripts. Another notable difference is that
% the ledger model presented in \cite{eutxoma} is a list of UTxO-stype transactions,
% rather than the UTxO set itself, with a unique initial state (the empty ledger).
% Here, we are not able to review and reason about the full transaction history,
% as is the case for existing realistic ledgers.


% Another
% aspirational aspect of automation of this project is that of the translation
% from contract implementations in Agda to Plutus, the language used in the
% between Agda contract implementations and Plutus contracts

\appendix
\section{Proofs}
\label{sec:appendix-proofs}

\paragraph{Proof of Lemma ($\MSGS$-payouts and DS). }
Suppose that $(\pi_{\fun{F,c}},~\pi_{\Tx},~\STRUC')$ is another (more permissive) structured contract,
with $\STRUC_{\MSGS} \subseteq \STRUC'_{\MSGS}$, and $\fun{s}~\STRUC_{\MSGS} = \fun{s}~\STRUC'_{\MSGS}$.
For any $(\emptytype, (s, m), tx, (s', m')) \in \STRUC'_{\MSGS}$, by definition,
\begin{align*}
  &\fun{getPayouts}_{\STRUC'}~(\emptytype, (s, m), tx, (s', m')) = \\
  &~~~~ \{~ms~\in~m' \setminus m~\mid~\fun{F}~(ms.\fun{msgFrom}) ~\wedge~ms.\fun{msgValue} > 0 \wedge
  \neg~(\fun{F}~(ms.\fun{msgTo}))~\}
\end{align*}

which depends only on $F$ (which is the same for $\STRUC$ and $\STRUC'$), and $m' \setminus m$.
Now, by the assumed preconditions on $\STRUC'$, for any $(\emptytype, (s, m), tx, (s', m')) \in \STRUC'_{\MSGS}$,
we can find $(\emptytype, (s, m), tx', (s', m')) \in \STRUC_{\MSGS} \subseteq \STRUC'_{\MSGS}$.
Then,
\begin{align*}
  \fun{C}~(\emptytype, (s, m), tx, (s' m'))~&=~\fun{C'}~(\fun{getPayouts}_{\STRUC'}~(\emptytype, (s, m), tx, (s', m'))) \\
  &=~\fun{C'}~(\fun{getPayouts}_{\STRUC}~(\emptytype, (s, m), tx', (s', m'))) \\
  &=~\fun{C}~(\emptytype, (s, m), tx', (s' m'))
\end{align*}

Therefore, any transition in $\STRUC'_{\MSGS}$ must also satisfy $\fun{C}$.
We get that $\STRUC'_{\MSGS} \cap \fun{C} = \STRUC'_{\MSGS}$, meaning
that $\STRUC_{\MSGS}$ is not vulnerable to DS with respect to such a $\fun{C}$.

\section{Pseudocode}
\label{sec:appendix-pseudocode}

\begin{figure}
  \begin{align*}
    \fun{msgOutsOK}& : \UTxO \to \B \\
    \fun{msgOutsOK}&~utxo \leteq~\\
    &~~~~~\forall~(i\mapsto o) \in utxo,
    \{~\msgsTT \mapsto \{ m \mapsto q \}\} \subseteq o.\val \Rightarrow\\
    &~~~~~~~~~~~~~~(q = 1) ~~\\
    &~~~~~~~~\wedge~~(m \neq \emptytype)~\wedge~(m.\fun{inUTxO} \mapsto \_ \notin utxo) ~~ \\
    &~~~~~~~~\wedge~~\applyMPScript{\msgsTT}~(\emptytype,~(i.\txrefid,~\msgsTT)) \\
    &~~~~~~~~\wedge~~\forall (i'\mapsto o') \in utxo, i\neq i',~\{~\msgsTT \mapsto \{ m \mapsto \_ \}\} \notin o'.\val \\
    &\wedge~\forall ~(tx, ix) \mapsto o~\in~utxo,~\forall ~i\in~tx.\inputs, \\
    &~~~~~~~~~~~~~~~~\applyScript{i.\fun{output}.\fun{validator}}~(i.\fun{output}.\datum,~i.\redeemer,~(tx,~i))~~ \\
    &~~~~~~~~~~\wedge~~(ix \mapsto o)\in~tx.\outputs
    \nextdef
    \SR &\leteq \{\send, \receive\} \\
    & \text{\emph{Tag specifying whether message is being sent or received}}
    \nextdef
    \fun{getMsgRef}& : \Msg \to (\OutputRef,~\Ix) \\
    \fun{getMsgRef}& ~\var{msg} \leteq (\var{msg}.\fun{inUTxO},~\var{msg}.\fun{msgIx}) \\
    & \text{\emph{Returns unique message identifier}}
  \end{align*}
\caption{Projections and auxiliary $\MSGS$ functions}
\label{fig:msgs-aux}
\end{figure}
%
\begin{figure}
  \begin{align*}
    & \msgsTT \leteq \msgsTT'~\msgsVal
    \nextdef
    &\applyScript{\msgsVal}~(\_,~\_,~(tx,~i))~\leteq \\
    &~~~~~\forall~\var{msg} \in~\{~m~\mid~\msgsTT'~(i.\fun{output}.\validator)~ \mapsto ~\{~m \mapsto 1 \} \} \subseteq ~i.\fun{output}.\val~\}, \\
    &~~~~~ \{~\msgsTT'~(i.\fun{output}.\validator)~ \mapsto ~\{ \var{msg} \mapsto -1 \} \} ~\subseteq~tx.\mint
  \end{align*}
\caption{Minting policy and validator for UTxO containing message tokens}
\label{fig:msgs-codeV}
\end{figure}
%
\begin{figure}
  \begin{align*}
    & \msgsTT' : \Script \to \Script \\
    &\applyMPScript{\msgsTT'~mv}~(\_,~(tx,~pid))~\leteq \\
    &~~~~~[~\fun{getMsgRef}~ m \mid (\_,\ m) \leftarrow \var{newOuts}~]~~\#~~
          [~\fun{getMsgRef}~ m \mid (\_,\ m) \leftarrow \var{usedInputs}~] \\
    &~~~~~\wedge~~\forall~(o,~\var{msg})~\in~\var{newOuts},~ \\
    &~~~~~~~~~~~~~~~~(\var{msg},~(\var{msg}.\fun{inUTxO}, \var{msg}.\fun{msgFrom}, \_))~\in~\var{sndMsgs} \\
    &~~~~~~~~~~\wedge~\{~t~ \subseteq~o.\val~\mid~\dom~t~=~\{pid\}~\} = \msgTkn~\var{msg} \\
    &~~~~~~~~~~\wedge~o.\validator~=~mv~\wedge~o.\val~\geq~\var{msg}.\fun{msgValue} \\
    &~~~~~\wedge~~\forall~(i,~\var{msg})~\in~\var{usedInputs},\
    (\var{msg},~(\_, \var{msg}.\fun{msgTo}, \_))~\in~\var{rcvMsgs}\\
    &~~~~~\wedge~~\Sigma_{(\_,\var{msg})~\in~\var{newOuts}}~\msgTkn~\var{msg} ~~+~~
    \Sigma_{(\_, \var{msg})~\in~\var{usedInputs}}~(-1) ~*~ (\msgTkn~\var{msg}) ~~=~~\\
    &~~~~~~~~~~~\{pid \mapsto tkns~\in~tx.\mint~\} \\
    &~~~~~\where \\
    &~~~~~~~~~\msgTkn~\var{msg}~\leteq \{~pid ~\mapsto~ \{\var{msg} \mapsto 1\}~\} \\
    &~~~~~~~~~\var{sndMsgs}~~~\leteq [~ (\var{msg},~i)~\mid~i\leftarrow~tx.\inputs,
    ~(sr,~\var{msg})~ \leftarrow~ i.\redeemer,~sr=\send~] \\
    &~~~~~~~~~\var{rcvMsgs}~~~~\leteq [~ (\var{msg},~i)~\mid~i\leftarrow~tx.\inputs,
    ~(sr,~\var{msg})~\leftarrow~i.\redeemer,~sr=\receive~] \\
    &~~~~~~~~~\var{newOuts}~~~\leteq \{~(o,~\var{msg})~\mid~o~\in~tx.\outputs,~
    \msgTkn~\var{msg}~\subseteq~o.\val~\} \\
    &~~~~~~~~~\var{usedInputs} \leteq \{~ (i,~\var{msg})~\mid~
    i~ \in~ tx.\inputs~tx,~\msgTkn~\var{msg}~\subseteq~i.\fun{output}.\val ~\} \\
  \end{align*}
\caption{Minting policy constructor for message tokens}
\label{fig:msgs-codeTT}
\end{figure}

\begin{figure}[htb]
\begin{subfigure}{.45\textwidth}
  \begin{align*}
    &\applyScript{\mathsf{checkMyFunction}}~(\_,~r,~(tx,~i))~\leteq~\\
    &~~~~~~~m.\fun{inUTxO}~=~i.\outputref \\
    &~~\wedge~m.\fun{msgFrom}~=~i.\fun{output} \\
    &~~\wedge~m.\fun{msgValue}~=~0 \\
    &~~\wedge~\msgTkn~m~\subseteq~tx.\mint \\
    &~~\wedge~\fun{myFunction}~\var{fIn}~=~\var{fOut} \\
    &~~\where \\
    &~~~~[(\send,~m)] ~=~r \\
    &~~~~(\var{fIn},~\var{fOut}) ~~=~m.\fun{msgData} \\
  \end{align*}
  \caption{Script minting message token.}
  \label{fig:checkmf}
\end{subfigure}
\hfill
\begin{subfigure}{.45\textwidth}
  \begin{align*}
    &\applyScript{\mathsf{useMyFunction}}~(d,~r,~(tx,~i))~\leteq~\\
    &~~~~~(~~~~m.\fun{msgFrom}~=~(\mathsf{checkMyFunction}, \_, \_)\\
    &~~~~\wedge~~ m.\fun{msgTo}~=~i.\fun{output} \\
    &~~~~\wedge~(-1)~*~(\msgTkn~m)~\subseteq~tx.\mint \\
    &~~~~\wedge~\fun{checkStuff}~d~r~~(tx,~i)~(\var{fIn},~\var{fOut})~~) \\
    &~~\vee~\fun{checkOtherStuff}~d~r~~(tx,~i)\\
    &~~\where \\
    &~~~~~~[(\receive,~~m)] ~=~r \\
    &~~~~~~(\var{fIn},~\var{fOut})~~~~~~~=~m.\fun{msgData} \\
  \end{align*}
\caption{Script using the memoized output.}
\label{fig:usemf}
\end{subfigure}
\caption{Scripts for memoizing the output of $\fun{myFunction}$.}
\label{fig:memo}
\end{figure}

%
%
%
% \section{First Section}
% \subsection{A Subsection Sample}
% Please note that the first paragraph of a section or subsection is
% not indented. The first paragraph that follows a table, figure,
% equation etc. does not need an indent, either.
%
% Subsequent paragraphs, however, are indented.
%
% \subsubsection{Sample Heading (Third Level)} Only two levels of
% headings should be numbered. Lower level headings remain unnumbered;
% they are formatted as run-in headings.
%
% \paragraph{Sample Heading (Fourth Level)}
% The contribution should contain no more than four levels of
% headings. Table~\ref{tab1} gives a summary of all heading levels.
%
% \begin{table}
% \caption{Table captions should be placed above the
% tables.}\label{tab1}
% \begin{tabular}{|l|l|l|}
% \hline
% Heading level &  Example & Font size and style\\
% \hline
% Title (centered) &  {\Large\bfseries Lecture Notes} & 14 point, bold\\
% 1st-level heading &  {\large\bfseries 1 Introduction} & 12 point, bold\\
% 2nd-level heading & {\bfseries 2.1 Printing Area} & 10 point, bold\\
% 3rd-level heading & {\bfseries Run-in Heading in Bold.} Text follows & 10 point, bold\\
% 4th-level heading & {\itshape Lowest Level Heading.} Text follows & 10 point, italic\\
% \hline
% \end{tabular}
% \end{table}
%
%
% \noindent Displayed equations are centered and set on a separate
% line.
% \begin{equation}
% x + y = z
% \end{equation}
% Please try to avoid rasterized images for line-art diagrams and
% schemas. Whenever possible, use vector graphics instead (see
% Fig.~\ref{fig1}).
%
% \begin{figure}
% \includegraphics[width=\textwidth]{fig1.eps}
% \caption{A figure caption is always placed below the illustration.
% Please note that short captions are centered, while long ones are
% justified by the macro package automatically.} \label{fig1}
% \end{figure}
%
% \begin{theorem}
% This is a sample theorem. The run-in heading is set in bold, while
% the following text appears in italics. Definitions, lemmas,
% propositions, and corollaries are styled the same way.
% \end{theorem}
%
% the environments 'definition', 'lemma', 'proposition', 'corollary',
% 'remark', and 'example' are defined in the LLNCS documentclass as well.
%
% \begin{proof}
% Proofs, examples, and remarks have the initial word in italics,
% while the following text appears in normal font.
% \end{proof}
% For citations of references, we prefer the use of square brackets
% and consecutive numbers. Citations using labels or the author/year
% convention are also acceptable. The following bibliography provides
% a sample reference list with entries for journal
% articles~\cite{ref_article1}, an LNCS chapter~\cite{ref_lncs1}, a
% book~\cite{ref_book1}, proceedings without editors~\cite{ref_proc1},
% and a homepage~\cite{ref_url1}. Multiple citations are grouped
% \cite{ref_article1,ref_lncs1,ref_book1},
% \cite{ref_article1,ref_book1,ref_proc1,ref_url1}.

% \subsubsection{Acknowledgements} Please place your acknowledgments at
% the end of the paper, preceded by an unnumbered run-in heading (i.e.
% 3rd-level heading).

%
% ---- Bibliography ----
%
% BibTeX users should specify bibliography style 'splncs04'.
% References will then be sorted and formatted in the correct style.
%
\bibliographystyle{splncs04}
\bibliography{msgs-bib}

\end{document}
