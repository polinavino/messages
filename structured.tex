\section{The Structured Contract Formalism}
\label{sec:struc}

The structured contract framework is a formalism for specifying and demonstrating
the correctness of the implementation of a stateful contract on $\LEDGER$.
A structured contract is specified in terms of small-steps semantics, then
a ledger reprsentation is given for its state and input types. The ledger representation
consists of
a pair of projection functions : (i) one which computes the contract state from
a given UTxO state (or fails), and (ii) one which computes the input to the contract
from the transaction being applied to the ledger. The environment (slot number)
is ignored because scripts are not permitted to inspect it in our model.

For any step in the $\LEDGER$, the representation functions must compute a
corresponding valid step in the structured contract any time the starting
UTxO state maps to a contract state (rather than fails).
This design guarantees that no invalid contract state updates are ever possible
on the ledger. Note that it is not necessarily true that for any contract
step, a corresponding valid ledger step exists. Specifying when one does exist
is the subject of future work.

\paragraph{Definition. }
\label{def:structured}

Suppose $\STRUC$ is small-step transition systems of the following type :
\begin{equation*}
  \_ \vdash
  \var{\_} \trans{struc}{\_} \var{\_}
  \in \powerset (\emptytype \times \State_{\STRUC} \times \Input_{\STRUC} \times \State_{\STRUC})
\end{equation*}

and let
\begin{align*}
  \pi_{\STRUC} : \UTxO \to \State_{\STRUC} \cup \emptytypeT \\
  \pi_{\Tx,\STRUC} : \Tx \to \Input_{\STRUC}
\end{align*}

be some projection functions such that the following implication holds :

  \begin{equation}
    \inference[$\sim >$]
    {
      \\~\\
      {
        \begin{array}{c}
          e\\
        \end{array}
      }
      \vdash
      {
        \left(
          \begin{array}{r}
            \var{u} \\
          \end{array}
        \right)
      }
      \trans{ledger}{\var{t}}
      {
        \left(
          \begin{array}{r}
            \var{u'} \\
          \end{array}
        \right)
      }
      \\~\\
    }
    {
      \begin{array}{r}
        * \\
      \end{array}
    \vdash
      (\pi_{\STRUC}~u)
      \trans{struc}{\pi_{\Tx,\STRUC}~t}
      (\varUpdate{\pi_{\STRUC}~u'})
    }
  \end{equation}

The triple $(\STRUC, \pi_{\STRUC}, \pi_{\Tx,\STRUC})$ is called a \emph{structured
contract}, and we denote it by
\[(\STRUC, \pi_{\STRUC}, \pi_{\Tx,\STRUC}) \succeq \LEDGER\]

Note that the range of the function $\pi_{\STRUC}$ is
$\State_{\STRUC} \cup \emptytypeT$, where $\emptytypeT$ is a
singleton. This indicates that $\pi_{\STRUC}$ is \emph{partial}. Note also that
the block-level data is never exposed to user-defined scripts in this model,
so that the context of a structured contract is necessarily of type $\{\emptytype\}$.

% \paragraph{Subsystems discussion. }
%
% We define a lockstep simulation-type subsystem relation here because it allows for the most uniform
% definition of a subsystem of a transition system. It also ensures that (i) each change observable within
% the subsystem is observable on the ledger, and (ii) it allows us to define a
% \emph{subsystem step} as an update to the subsystem done by a transaction.
% A lockstep subsystem does not require separating inputs into those that affect
% the state, and those that do not, and treating them differently.
%
% This means that \emph{most} subsystem will include
% a trivial step. This is a deliberate choice to ensure all behaviour of the subsystem transition
% is made explicit at the specification level. In particular, a trivial step may
% have specific predicates that disambiguate it from non-trivial steps in a (possibly)
% non-deterministic specification. On the other hand, a trivial step in the
% larger system may not always have to have a corresponding step in the subsystem,
% hence the $u \neq u'$ constraint.
%
% % Note here that, given a contract state $s$ such that $\var{utxo} \sim s$,
% % the only entries in the UTxO set are ones that contain some part of the
% % representation of the state $s$. Then, any transaction applied to $\var{utxo}$
% % will necessarily change some part of the state representation, and therefore
% % correspond to a non-trivial state update.
%
% Note that \ref{def:subsystem} includes a \emph{proof obligation} that must be
% fulfilled as part of specifying the subsystem, similar to how defining a
% a \emph{group} requires a proof of the associativity of a group action.
% As it is an obligation, and \emph{does not define a rule}, one can construct
% plenty of transition
% systems with relations for which this proof obligation cannot
% be fulfilled. However, \emph{if} it is possible, $\STRUC$ is a subsystem of $\TRANS$.
%
% Note that subsystem relation is transitive and reflexive.
% The symmetric property
% does not hold for many of the examples of interest.
% Whenever $\TRANS~=~\STRUC$,
% we say that the two transition systems \emph{are equivalent}. Note here that
% this is an equality of relations, indicating that they have the same membership, but
% the rules specifying that membership for both relations may be different.
%
% Note also that an arbitrary transition $\mathsf{TRANS}$ may not be a function
% of the first three variables,
% and may well allow two distinct final states for a given initial state, environment,
% and input. However, the $\mathsf{LEDGER}$ rule is indeed functional, as the
% final state variable $\var{utxo'}$ in its only rule, $\mathsf{ApplyTx}$,
% is defined as a function of the other values in the precondition.
%
% \subsection{Structured contracts. }
% \label{sec:struc-def}
%
% Our subsystem definition is general, however, the rest of this work is
% geared towards reasoning about the programmable
% parts of the ledger, ie. those where the permissions are controlled by user-defined
% scripts. For this reason, we define a particular class of subsystems.
%
% A \emph{structured contract} $\STRUC$ is a transition system subsystem of the $\LEDGER$, where
% the $\Env$ of $\STRUC$ is $\emptytype$, and
% for some functions $\pi_{\STRUC} : \UTxO \to \State_{\STRUC}^?$ and $\pi_{\Tx,\STRUC} : \Tx \to \Input_{\STRUC}$, the relations $\sim$ and $\sim >$ are given by :
%
% \begin{align*}
%   \var{utxo}~\sim~s &\leteq (\pi_{\STRUC}~\var{utxo} ~=~s)
%   \nextdef
%   (\var{slot},~\var{tx})~\sim >~ (\emptytype, i) &\leteq (\pi_{\Tx,\STRUC}~\var{tx} = i)
% \end{align*}
%
% As shorthand, we denote the subsystem relation of structured contracts as
%
% \[ (\STRUC,~\pi_{\STRUC},~\pi_{\Tx,\STRUC}) \succeq \LEDGER\]
%
% We also sometimes denote both $\pi_{\Tx,\STRUC}$ and $\pi_{\STRUC}$
% by $\pi$ or $\pi_{\Tx}$ when the context is clear. Note here that $\LEDGER$
% has no trivial transitions due to the fact that transactions without
% inputs are not allowed by rule \ref{rule:has-input}, so the UTxO set is necessarily updated
% by the transaction.
%
% The purpose of structured contracts is to represent all ledger
% subsystems that are programmable by the user via scripts.
% The reason that the environment of the structured contract is empty is that no
% environment data from $\LEDGER$ is ever available to any scripts used to implement contracts.
% In particular, scripts are not able to inspect the current $\var{slot}$ number.
% It is possible that $\LEDGER$ is also a subsystem of $\STRUC$, but this is by no means
% a requirement. We given an example of when this is the case and how we can demonstrate
% this in \ref{sec:toggle}, however, specifying the class of structured contracts
% for which this is necessarily the case is the subject of future work.
%
% The reason that the $\pi_{\State}$ function is partial is that we may not want to
% relate certain ledger states to specification states, such as ones where
% the structured contract UTxO representation is either uninitialized
% or cannot be initialized. However, any ledger that that is
% related to a $\STRUC$ state can only ever be updated by a transaction to another
% state that is related to a $\STRUC$ state, and any such update must always correspond
% to a valid $\STRUC$ update.
%
% \subsection{Preservation of Value example}
% \label{sec:pov}
%
% The EUTxO ledger transition system $\LEDGER$ is said to satisfy the
% \emph{preservation of value (POV)} property. A \emph{local} POV is
% the property that each valid transaction mints exactly the difference
% between the value in the UTxOs in spends and those it creates.
% This is ensured by the ledger rule \ref{rule:value-is-preserved}, which is
%
% \[ \mint~\var{tx} + \sum_{i \in \inputs~\var{tx},~(\outputref~i)\mapsto~o~\in~\var{utxo}} \val~\var{o} = \sum_{o \in \outputs~\var{tx}} \val~\var{o} \]
%
% The \emph{global} POV follows from the local POV and is the property we will express using the
% structured contract formalism. It states that the sum total of all the tokens
% recorded on the ledger is changed by exactly the amount minted or burned by the
% transaction,
%
% \[ \sum_{\var{or}\mapsto \var{out}\in \var{utxo'}}~\val~\var{out} ~= ~(\sum_{\var{or}\mapsto \var{out}\in \var{utxo}}~\val~\var{out}) ~+~(\fun{mint}~ \var{tx}) \]
%
% where any $(\var{s},~\var{utxo},~\var{tx},~\var{utxo'}) \in \LEDGER$,
% and (recall that) the updated UTxO set is defined as follows
%
% \[ \var{utxo'}~\leteq~(\fun{txins}~{tx} \subtractdom \var{utxo}) \cup \fun{mkOuts}~{tx} \]
%
% We can define a subsystem $\mathsf{POV}$ of $\LEDGER$ that specifies how the total value of
% all assets on the ledger is updated :
%
% % \begin{figure}[htb]
%   \emph{Transition type}
%   \begin{equation*}
%     \_ \vdash
%     \var{\_} \trans{pov}{\_} \var{\_}
%     \subseteq \powerset (\emptytype \times \Value \times \Tx \times \Value)
%   \end{equation*}
%   %
%   \emph{Subsystem relations}
%   \begin{align*}
%     \pi_{\Tx,\POV}~\var{tx}~ ~&\leteq~\var{tx}
%     \nextdef
%     \pi_{\POV}~\var{utxo}~&\leteq~\sum_{\wcard \mapsto \var{out}\in \var{utxo}}~\val~\var{out}
%   \end{align*}
%   %
%   \emph{Transition rule}
%   \begin{equation}
%     \inference[UpdateValTotal]
%     {
%     i \leteq \fun{mint}~\var{tx}
%     \\ ~ \\
%     }
%     {
%     \begin{array}{l}
%       \\
%     \end{array}
%       \vdash
%       \left(
%       \begin{array}{r}
%         \var{s} \\
%       \end{array}
%       \right)
%       \trans{pov}{tx}
%       \left(
%       \begin{array}{r}
%         \varUpdate{\var{s}~+~\var{i}}  \\
%       \end{array}
%       \right) \\
%     }
%   \end{equation}
% %   \caption{Specification of the $\mathsf{UpdateValTotal}$ POV subsystem}
% %   \label{fig:pov-spec}
% % \end{figure}
%
% \begin{equation}
%   \inference[UpdatemyNFTTotal]
%   {
%   i \leteq \{~\fun{myNFTPolicy}~\mapsto~\var{tkns}~\in~~\fun{mint}~\var{tx}~\} \\~\\
%   0~\leq~s~\leq~s~+~i~\leq~\{~\fun{myNFTPolicy}~\mapsto~(""~ \mapsto~1)~\}
%   \\ ~ \\
%   }
%   {
%   \begin{array}{l}
%     \\
%   \end{array}
%     \vdash
%     \left(
%     \begin{array}{r}
%       \var{s} \\
%     \end{array}
%     \right)
%     \trans{nft}{tx}
%     \left(
%     \begin{array}{r}
%       \varUpdate{\var{s}~+~\var{i}}  \\
%     \end{array}
%     \right) \\
%   }
% \end{equation}
%
% The value is
% changed exactly by the amount specified in the mint field, so that the subsystem proof
% obligation corresponds exactly to the global POV formulation.
% To complete the definition, it remains to prove
% the subsystem relation, which requires that, given
%
% \begin{align*}
%   (\var{slot},~\var{utxo},~\var{tx},~\var{utxo'}) ~\in \LEDGER
% \end{align*}
%
% necessarily
%
% \begin{align*}
%   (\emptytype,~\pi~\var{utxo},~\pi~\var{tx},~\pi~\var{utxo'}) \in \POV
% \end{align*}
%
% So, we must show that
%
% \[ \sum_{\wcard \mapsto \var{out}\in \var{utxo'}}~\val~\var{out} = \sum_{\wcard \mapsto \var{out}\in \var{utxo}}~\val~\var{out} + \fun{mint}~\var{tx} \]
%
% The proof relies on the (local) POV $\LEDGER$ rule :
%
% % \begin{ruledfigure}{t}
%   \begin{align*}
%     \sum_{\var{or}\mapsto \var{out}\in \var{utxo'}}~\val~\var{out}
%     & = ~\sum_{\var{or}\mapsto \var{out}\in ((\fun{getORefs}~\var{tx}) \subtractdom \var{utxo}) \cup \fun{mkOuts}~{tx}}~\val~\var{out} \\
%     & = ~\sum_{\var{or}\mapsto \var{out}\in \var{utxo}}~\val~\var{out}
%     ~-~\sum_{i \in \inputs~\var{tx},~(\outputref~i\mapsto~\var{out})\in \var{utxo}}~\val~\var{out}
%     ~+~\sum_{\var{out} \in \outputs ~\var{tx}}~\val~\var{out} \\
%     & = ~\sum_{\var{or}\mapsto \var{out}\in \var{utxo}}~\val~\var{out}
%     ~-~\sum_{i \in \inputs~\var{tx},~(\outputref~i\mapsto~\var{out})\in \var{utxo}}~\val~\var{out}
%     ~+~(\mint~\var{tx}) \\
%     &~~~~~~~ + \sum_{i \in \inputs~\var{tx},~(\outputref~i\mapsto~\var{out})\in \var{utxo}}~\val~\var{out} \\
%     & = ~\sum_{\var{or}\mapsto \var{out}\in \var{utxo}}~\val~\var{out}  ~+~(\mint~\var{tx})
%   \end{align*}
% % \caption{Proof of the $\mathsf{Simulate}$ constraint for POV}
% % \label{fig:pov-pf}
% % \end{ruledfigure}
%
% Now, we observe that the
% statement that $\mathsf{POV}$ is a subsystem of the ledger is exactly the statement that the
% preservation of value property of $\LEDGER$ holds.
% This gives a simple but illustrative example of the power of the subsystem definition
% to use local properties to derive global invariants.
%
% % Prove the POV property for the subsystem
% % Automatically derive the POV property for the ledger
% % While the workflow of Fig.9 is:
% % Prove an ad-hoc property for the subsystem (not really formulated as the POV property of the subsystem)
% % Wow! It turns out it's exactly identical to the POV property for the ledger, QED.
%
%
% \paragraph{NFT example.}
% \label{sec:nft-example}
%
% We specify
% a structured contract keeping track of the total amount of a specific (unburnable)
% NFT token.
% The defining feature of this contract is that the total amount of this token in
% existence should always be either 1 or 0, and it cannot be burned.
% We can define the contract specification relation that only add up tokens under
% policy $\fun{myNFTPolicy}$ with the following changes to the definition of the
% state projection function :
%
% \begin{align*}
%   & \pi~\var{utxo}~ \leteq~\begin{cases}
%     0 & \text{ if } \fun{hasRef}~~\wedge~~s = 0 \\
%     s & \text{ if } s = \{ ~\fun{myNFTPolicy}~\mapsto~\{""~ \mapsto~1\} ~\} ~~\wedge~~ \neg~\fun{hasRef}\\
%     \emptytype & \text{ otherwise } \\
%   \end{cases} \\
%   &  \where \\
%   &  ~~~~s \leteq~ \{~p~\mapsto~ \var{tkns}~\mid~p\mapsto \var{tkns} \in \sum_{\wcard \mapsto \var{out}\in \var{utxo}}~\val~\var{out},~p = \fun{myNFTPolicy}~\} \\
%   &  ~~~~\fun{hasRef} \leteq (\fun{myNFTRef} \mapsto (\fun{myNFTScript},~\wcard,~d) \in \var{utxo})
% \end{align*}
%
% Here, $\pi~\var{utxo}$ returns a non-$\emptytype$ result when either (i) the output
% $\fun{myNFTRef} \mapsto (\fun{myNFTScript},~\wcard,~d)$ is in the UTxO set and no
% NFTs under the policy $\fun{myNFTPolicy}$ have been minted yet, or (ii) the NFT under
% the $\fun{myNFTPolicy}$ policy is already minted with the correct token name $""$
% and quantity $1$, and there is no $\fun{myNFTRef}$ in the UTxO set output references.
%
%
% This example differs from the general POV contract example in a
% noteworthy way. The POV contract represents a property of the ledger transition system.
% Constructing instances of POV contracts for tokens under specific policies, on the
% other hand, is a tool for developing and verifying reliable minting policy code that behaves in
% the specified way. It can be used to define and verify properties of the contract.
% A sensible approach is to first leave the scripts in the specification (i.e.
% $\fun{myNFTPolicy}$ and $\fun{myNFTScript}$) unspecified, and define the
% specification update rules as function of the contract state token total,
% as well as the input transaction.
% Then, compose the unspecified scripts such that it possible to prove the required subsystem
% relation $\sim >$.
